%\section{Приложение}
%
%    \subsection{Задание №2}
%
%        Необходимые условия экстремума функции $f \left( x, y, \ldots)$ в точке $A$ заключаются в выполнении в этой точке равенств: $\frac{\partial f}{\partial x} = 0, \: \frac{\partial f}{\partial x} = 0, \: \ldots$. При этом функция двух переменных $z = f \left( x, y)$ имеет в данной точке максимум, если $\triangle = \frac{\partial^{2} f}{\partial x^{2}} \frac{\partial^{2} f}{\partial y^{2}} - \left( \frac{\partial^{2} f}{\partial x \partial y} \right)^2 > 0$ и $\frac{\partial^{2} f}{\partial x^{2}}$ или $\frac{\partial^{2} f}{\partial y^{2}} < 0$, и минимум, если $\triangle > 0$ и $\frac{\partial^{2} f}{\partial x^{2}}$ или $\frac{\partial^{2} f}{\partial y^{2}} > 0$ (при условии непрерывности частных производных).\\
%
%    \subsection{Задание №7}
%
%        Если функция $f \left( x)$ задана на сегменте $[-l,l]$, где $l$ -- произвольное число, то при выполнении на этом сегменте условий Дирихле указанная функция может быть представлена в виде суммы ряда Фурье:
%
%        \begin{equation}
%            \frac{a_{0}}{2} + \sum_{m = 1}^{\infty} \left( a_{m} \cos \frac{m \pi x}{l} + b_{m} \sin \frac{m \pi x}{l} \right),
%        \end{equation}
%
%        где
%
%        \begin{equation}
%            a_{0} =  \frac{1}{l} \int_{-l}^{l} f \left( x) dx, \qquad a_{m} = \frac{1}{l} \int_{-l}^{l} f \left( x) \cos \frac{m \pi x}{l} dx, \qquad b_{m} = \frac{1}{l} \int_{-l}^{l} f \left( x) \sin \frac{m \pi x}{l} dx.
%        \end{equation}

\section*{Теория}

    \subsection*{Квадратичные формы}

    	Квадратичной формой действительных переменных $x_{1}, x_{2}, \ldots, x_{n}$ называется многочлен второй степени относительно этих переменных, не содержащий свободного члена и членов первой степени. \\

    	Если $n = 2$, то 

    	\begin{equation}
    		f \left( x_{1}, x_{2} \right) = a_{11} x^{2}_{1} + a_{12} x_{1} x_{2} + a_{22} x^{2}_{2}
    	\end{equation}

    	Если $n = 3$, то 

    	\begin{equation}
    		f \left( x_{1}, x_{2}, x_{3} \right) = a_{11} x^{2}_{1} + a_{22} x^{2}_{2} + a_{33} x^{2}_{2} + 2 a_{12} x_{1} x_{2} + 2 a_{13} x_{1} x_{3} + 2 a_{23} x_{2} x_{3}
    	\end{equation}

    	В дальнейшем все необходимые формулировки и определения приведем для квадратичной формы трех переменных.

    	Матрица

    	\begin{equation}
    		\mathbf{A} =
    		\begin{bmatrix}
    			a_{11} & a_{12} & a_{13} \\
    			a_{21} & a_{22} & a_{23} \\
    			a_{31} & a_{32} & a_{33}
    		\end{bmatrix}
    	\end{equation}

    	у которой $a_{ik} = a_{ki}$, называется матрицей квадратичной формы $f \left( x_{1}, x_{2}, x_{3} \right)$, а соответствующий определитель - определителем этой квадратичной формы.

    	Так как $\mathbf{A}$ - симметрическая матрица, то корни $\lambda_{i}$ характеристического уравнения

    	\begin{equation}
    		\det \left( \mathbf{A} - \lambda \mathbf{I} \right) = 0
    	\end{equation}

    	являются действительными числами.

    	Пусть $\mathbf{u}_{i}$ нормированные собственные векторы, соответствующие характеристическим числам $\lambda_{i}$. Векторы исходной системы координат - $\mathbf{e}_{i}$.

    	Матрица

    	\begin{equation}
    		\mathbf{B} = \left[ \mathbf{e}_{i} \cdot \mathbf{u}_{j} \right]
    	\end{equation}

        является матрицей перехода от базиса $\mathbf{e}_{i}$ к базису $\mathbf{u}_{i}$. Матрица $\mathbf{A}$ в новой системе координат

        \begin{equation}
            \tilde{\mathbf{A}} = \mathbf{B A B}^{T}
        \end{equation}

        Формулы преобразования координат при переходе к новому ортонормированному базису имеют вид

        \begin{equation}
            \mathbf{x} = \tilde{\mathbf{A}} \tilde{\mathbf{x}}
        \end{equation}

        Преобразовав с помощью этих формул квадратичную форму $f \left( x_{1}, x_{2}, x_{3} \right)$, получаем квадратичную форму

        \begin{equation}
            f \left( \tilde{x}_{1}, \tilde{x}_{2}, \tilde{x}_{3} \right) = a_{11} \tilde{x}^{2}_{1} + a_{22} \tilde{x}^{2}_{2} + a_{33} \tilde{x}^{2}_{2}
        \end{equation}

        не содержащую членов с произведениями $\tilde{x}_{1} \tilde{x}_{2}$, $\tilde{x}_{1} \tilde{x}_{3}$, $\tilde{x}_{2} \tilde{x}_{3}$.

        Принято говорить, что квадратичная форма $f \left( x_{1}, x_{2}, x_{3} \right)$ приведена к каноническому виду с помощью преобразования $\tilde{\mathbf{A}}$.

		\newpage

    \subsection*{Экстремум функции нескольких переменных}

	    \subsubsection*{Определение экстремума}

		    Пусть функция $f \left( P \right) = f \left( x_{1}, \ldots, x_{n} \right)$ определена в окрестности точки $P_{0}$. Если или $f \left( P_{0} \right) > f \left( P \right)$, или $f \left( P_{0} \right) < f \left( P \right)$ при $0 < \rho \left( P_{0}, P \right) < \delta$, то говорят, что функция $f \left( P \right)$ имеет строгий \textit{экстремум} (соответственно \textit{максимум} или \textit{минимум}) в точке $P_{0}$.\\

	    \subsubsection*{Необходимое условие экстремума}

		    Дифференцируемая функция $f \left( P \right)$ может достигать экстремума лишь в \textit{стационарной} точке $P_{0}$, т.е. такой, что $d f \left( P_{0} \right) = 0$. Следовательно, точки экстремума функции $f \left( P \right)$ удовлетворяют системе уравнений $f_{x_{i}} \left( x_{1}, \ldots, x_{n} \right) = 0 \: (i = 1, \ldots, n)$.\\

	    \subsubsection*{Достаточное условие экстремума}

		    Функция $f \left( P \right)$ в точке $P_{0}$ имеет:

		    \begin{itemize}
		    	\item \textit{максимум}, если $d f \left( P_{0} \right), \: d^{2} f \left( P_{0} \right) < 0$, при $\sum^{n}_{i = 1} \left| d x_{i} \right| \neq 0$.
		    	\item \textit{минимум}, если $d f \left( P_{0} \right), \: d^{2} f \left( P_{0} \right) > 0$, при $\sum^{n}_{i = 1} \left| d x_{i} \right| \neq 0$.
		    \end{itemize}

		    Исследование знака второго дифференциала $d^{2} f \left( P_{0} \right)$ может быть проведено путем приведения соответствующей квадратичной формы к каноническому виду.\\

		    В частности, для случая функции $f \left( x, y \right)$ двух независимых переменных $x$ и $y$ в стационарной точке $\left( x_{0}, y_{0} \right) \left( d f \left( x_{0}, y_{0} \right) =0 \right)$ при условии, что $D = A C - B^{2}$, где $A = f_{xx} \left( x_{0}, y_{0} \right), B = f_{xy} \left( x_{0}, y_{0} \right), C = f_{yy} \left( x_{0}, y_{0} \right)$ имеем:

		    \begin{enumerate}
		    	\item \textit{минимум}, если $D > 0, \: A > 0 \: \left( C > 0 \right)$;
		    	\item \textit{максимум}, если $D > 0, \: A < 0 \: \left( C < 0 \right)$;
		    	\item \textit{отсутствие экстремума}, если $D < 0$.
		    \end{enumerate}

	    \subsubsection*{Условный экстремум}

		    Задача определния экстремума функции $f \left( P \right) = f \left( x_{1}, \ldots, x_{n} \right)$ при наличии ряда соотношений $\varphi_{i} \left( P \right) \: \left( i = 1, \ldots, m; \: m < n \right)$ сводится к нахождению обычного экстремума для \textit{функции Лагранжа}

		    \begin{equation}
		        L \left( P \right) = f \left( P \right) + \sum^{m}_{i = 1} \lambda_{i} \varphi_{i} \left( P \right),
		    \end{equation}

		    где $\lambda_{i} \: \left( i = 1, \ldots, m \right)$ -- постоянные множители. Вопрос о существовании и характере условного экстремума в простейшем случае решается на основании исследования знака второго дифференциала $d^{2} L \left( P_{0} \right)$ в стационарной точке $\left( P_{0} \right)$ функции $L \left( P \right)$ при условии, что переменные $d x_{1}, \ldots, d x_{n}$ связаны соотношениями

		    \begin{equation}
		        \sum^{n}_{j = 1} \frac{\partial \varphi_{i}}{\partial x_{j}} d x_{j} = 0 \: \left( i = 1, \ldots, m \right).
		    \end{equation}

		\subsubsection*{Абсоютный экстремум}

			Функция $f \left( P \right)$, дифференцируемая в ограниченной и замкнутой области, достигает своих наибольшего и наименьшего значений в этой области или в стационарной точке, или в граничной точке области.

%		    Необходимые условия экстремума функции $f \left( x, y, \ldots)$ в точке $A$ заключаются в выполнении в этой точке равенств: $\frac{\partial f}{\partial x} = 0, \: \frac{\partial f}{\partial x} = 0, \: \ldots$. При этом функция двух переменных $z = f \left( x, y)$ имеет в данной точке максимум, если $\triangle = \frac{\partial^{2} f}{\partial x^{2}} \frac{\partial^{2} f}{\partial y^{2}} - \left( \frac{\partial^{2} f}{\partial x \partial y} \right)^2 > 0$ и $\frac{\partial^{2} f}{\partial x^{2}}$ или $\frac{\partial^{2} f}{\partial y^{2}} < 0$, и минимум, если $\triangle > 0$ и $\frac{\partial^{2} f}{\partial x^{2}}$ или $\frac{\partial^{2} f}{\partial y^{2}} > 0$ (при условии непрерывности частных производных).

		\newpage

	\subsection*{Исследование функций}

	    \subsubsection*{Схема элементарного исследования графика функции}

	        \begin{enumerate}
	        	\item Область определения;

	        	\item Область значений;

	        	\item Четность, нечетность функции;

	        	\item Характерные точки:

	        	\begin{enumerate}
	        		\item точки пересечения графика с осями;

	        		\item предельные значения функции;

	        		\item экстремальные значения;

	        		\item точки перегиба др.
	        	\end{enumerate}

	        	\item Асимптоты;

	        	\item Построение графика.
	        \end{enumerate}

			\paragraph{Экстремум}

			    $f' (x) = 0 \Rightarrow x^{0}_{i} \Rightarrow f'' (x^{0}_{i}) < 0 - \max$, $f'' (x^{0}_{i}) > 0 - \min$; если не обращающаяся в нуль первая производная четного порядка, то при $f^{(2 k)} (x^{0}_{i}) > 0 - \min$ и $f^{(2 k)} (x^{0}_{i}) < 0 - \max$; если $f^{(2 k + 1)} (x^{0}_{i}) \neq 0$, то перегиб.

		    \paragraph{Точка перегиба}

			    Если $f'' (x^{0}_{i}) > 0$, то функция локально выпукла вниз, если $f'' (x^{0}_{i}) < 0$, то локально вверх.

		    \paragraph{Асимптоты}

			    Кривая $y = f \left( x \right)$ имеет горизонтальную асимптоту $y = b$, если

			    \begin{equation}
			        \lim_{x \rightarrow \pm \infty} f \left( x \right) = b,
			    \end{equation}

				вертикальную асимптоту $x = a$, если при $x \rightarrow a$, или $x \rightarrow a + 0$, или $x \rightarrow a - 0$

				\begin{equation}
					\lim_{x \rightarrow a} f \left( x \right) = \pm \infty,
				\end{equation}

				\begin{equation}
					\lim_{x \rightarrow a + 0} f \left( x \right) = \pm \infty,
				\end{equation}

				\begin{equation}
					\lim_{x \rightarrow a - 0} f \left( x \right)) = \pm \infty,
				\end{equation}

				наклонную асимптоту $y = k x + b$, если

				\begin{equation}
					k_{1} = \lim_{x \rightarrow + \infty} \frac{f \left( x \right)}{x} \: \mbox{или} \: k_{2} = \lim_{x \rightarrow - \infty} \frac{f \left( x \right)}{x}
				\end{equation}

				\begin{equation}
					b_{1} = \lim_{x \rightarrow + \infty} \left[ f \left( x \right) - k_{1} x \right] \: \mbox{или} \: b_{2} = \lim_{x \rightarrow - \infty} \left[ f \left( x \right) - k_{2} x \right].
				\end{equation}

		\subsubsection*{Графики неявно заданных функций}

			Пусть функция $y = y \left( x \right)$ определяется неявно уравнением $F \left( x, y \right) = 0$.\\

			\paragraph{Общие свойства}

				\begin{enumerate}
					\item Если $F \left( x, y \right) = F \left( - x, y \right)$, то кривая симметрична относительно \textbf{OY};

					\item Если $F \left( x, y \right) = F \left( x, - y \right)$, то кривая симметрична относительно \textbf{OX};

					\item Если $F \left( x, y \right) = F \left( - x, - y \right)$, то кривая симметрична относительно $\left( 0, 0 \right)$;

					\item Если $F \left( x, y \right) = F \left( y, x \right)$, то кривая симметрична относительно биссектрисы $y = x$;

					\item График $F \left( x + a, y \right) = 0$ получается из  путем переноса последнего на $\left| a \right|$ по оси \textbf{OX};

					\item График $F \left( x, y  + b \right) = 0$ получается из  путем переноса последнего на $\left| b \right|$ по оси \textbf{OY};
	
					\item График $F \left( \frac{x}{p}, y \right)$ получается из $F \left( x, y \right)$  путем растяжения в $p$ раз по оси \textbf{OX};

					\item График $F \left( x, \frac{y}{q} \right)$ получается из $F \left( x, y \right)$  путем растяжения в $q$ раз по оси \textbf{OY};
				\end{enumerate}

			\paragraph{Точки пересечения кривой $F \left( x, y \right)$ с осями координат}

				\begin{enumerate}
					\item С осью \textbf{OX}:

						\begin{equation}
							\begin{cases}
								& F \left( x, 0 \right) = 0\\
								& F \left( x, y \right) = 0
							\end{cases};
						\end{equation}

					\item С осью \textbf{OY}:

						\begin{equation}
							\begin{cases}
								& F \left( 0, y \right) = 0\\
								& F \left( x, y \right) = 0
							\end{cases}.
						\end{equation}

				\end{enumerate}

			\paragraph{Асимптоты кривой $F \left( x, y \right)$}

				\begin{enumerate}
					\item Горизонтальная асимптота: приравниваем нулю коэффициент при старшей степени $x$, если этот коэффициент постоянен, то горизонтальных асимптот нет;
					
					\item Вертикальная асимптота: приравниваем нулю коэффициент при старшей степени $y$, если он постоянен, то вертикальных асимптот нет;
					
					\item Наклонная асимптота: заменяем $y$ в уравнении $F \left( x, y \right)$ на $y = k x + b$, затем в уравнении  приравниваем два коэффициента при старших степенях $x$. Решаем полученную систему и находим $k$ и $b$.
				\end{enumerate}

			\paragraph{Особые точки кривой $F \left( x, y \right)$}

				Точка  кривой $M \left( x_{0}, y_{0} \right)$ называется особой, если ее координаты одновременно удовлетворяют трем равенствам:

				\begin{equation}
					\begin{cases}
						& F \left( x_{0}, y_{0} \right) = 0\\
						& F_{x} \left( x_{0}, y_{0} \right) = 0\\
						& F_{y} \left( x_{0}, y_{0} \right) = 0
					\end{cases}.
				\end{equation}

				Угловой коэффициент касательной в особой точке (угол входа)

				\begin{equation}
					k = - \frac{F_{x}}{F_{y}}
				\end{equation}

				неопределенный.\\

				Предположим, что в особой точке $M \left( x_{0}, y_{0} \right)$ $\left| F_{xx} \right| + \left| F_{xy} \right| + \left| F_{yy} \right| \neq 0$ т.е. не все частные производные второго порядка обращаются в нуль. Тогда $M \left( x_{0}, y_{0} \right)$ -- двойная точка.\\

				Введем характеристику двойной точки.

				\begin{equation}
					\vartriangle = F^{2}_{xy} \left( x_{0}, y_{0} \right) - F_{xx} \left( x_{0}, y_{0} \right) F_{yy} \left( x_{0}, y_{0} \right).
				\end{equation}

				Тогда:

				\begin{enumerate}
					\item $\vartriangle > 0$ -- кривая имеет узловую точку;

					\item $\vartriangle < 0$ -- изолированная точка;

					\item $\vartriangle = 0$ -- тогда точка $M \left( x_{0}, y_{0} \right)$ может быть:

					\begin{enumerate}
						\item точкой возврата первого рода;

						\item точкой возврата второго рода;

						\item изолированной точкой;

						\item точкой самокасания. 
					\end{enumerate}
				\end{enumerate}

			\paragraph{Точки экстремума}

				Для определения точек подозрительных на экстремум нужно  воспользоваться представлением для углового коэффициента:

				\begin{equation}
					k = - \frac{F_{x}}{F_{y}}.
				\end{equation}

				Теперь, чтобы найти на кривой точку, где касательная параллельна \textbf{OX}, надо решить систему

				\begin{equation}
					\begin{cases}
						& F \left( x, y \right) = 0\\
						& F_{x} \left( x, y \right) = 0
					\end{cases}.
				\end{equation}

				Пусть $\left( x_{1}, y_{1} \right)$ -- ее корни, причем $F'_{y} \left( x_{0}, y_{0} \right) \neq 0$, тогда $M \left( x_{1}, y_{1} \right)$ для кривой $F \left( x_{0}, y_{0} \right) = 0$ будет:

				\begin{itemize}
					\item точкой $Y_{\max}$, если $F_{xx} \left( x_{1}, y_{1} \right) F_{y} \left( x_{1}, y_{1} \right) > 0$;

					\item точкой $Y_{\min}$, если $F_{xx} \left( x_{1}, y_{1} \right) F_{y} \left( x_{1}, y_{1} \right) < 0$.
				\end{itemize}

				Если требуется найти точки на кривой, где касательная параллельна \textbf{OY}, нужно решить систему

				\begin{equation}
					\begin{cases}
						& F \left( x, y \right) = 0\\
						& F_{y} \left( x, y \right) = 0
					\end{cases}.
				\end{equation}

				Точка $M \left( x_{2}, y_{2} \right)$ на кривой $F \left( x, y \right) = 0$ будет:

				\begin{itemize}
					\item точкой $X_{\max}$, если $F_{yy} \left( x_{2}, y_{2} \right) F_{y} \left( x_{2}, y_{2} \right) > 0$;
					
					\item точкой $X_{\min}$, если $F_{yy} \left( x_{2}, y_{2} \right) F_{y} \left( x_{2}, y_{2} \right) < 0$.
				\end{itemize} 

			\paragraph{Точки перегиба}

				Если уравнение $F \left( x, y \right) = 0$ нельзя явно разрешить относительно $y$, то найти точки перегиба очень трудно.\\

				Для алгебраической кривой $F \left( x, y \right) = 0$ точки перегиба ее находятся в местах пересечения кривой и ее гессианы:

				\begin{equation}
					\begin{cases}
						& F \left( x, y \right) = 0\\
						& F_{xx} F^{2}_{y} - 2 F_{xy} F_{x} F_{y} + F_{yy} F^{2}_{x} = 0
					\end{cases}.
				\end{equation}

		\newpage

	\subsection*{Ряд Фурье}

		Если функция $f \left( x \right)$ задана на сегменте $[-l,l]$, где $l$ -- произвольное число, то при выполнении на этом сегменте условий Дирихле (см., к примеру, википедию) указанная функция может быть представлена в виде суммы ряда Фурье:

		\begin{equation}
			f \left( x \right) = \frac{a_{0}}{2} + \sum_{m = 1}^{\infty} \left( a_{m} \cos \frac{m \pi x}{l} + b_{m} \sin \frac{m \pi x}{l} \right),
		\end{equation}

		где

		\begin{equation}
			a_{0} =  \frac{1}{l} \int_{-l}^{l} f \left( x \right) dx, \; a_{m} = \frac{1}{l} \int_{-l}^{l} f \left( x \right) \cos \frac{m \pi x}{l} dx, \; b_{m} = \frac{1}{l} \int_{-l}^{l} f \left( x \right) \sin \frac{m \pi x}{l} dx.
		\end{equation}

		\newpage

	\subsection*{Приведение к каноническому виду уравнения второго порядка с двумя независимыми переменными}

	    Рассмотрим уравнение второго порядка с двумя независимыми переменными

	    \begin{equation}
	        A \left( x, y \right) u_{xx} + 2 B \left( x, y \right) u_{xy} + C \left( x, y \right) u_{yy} + F \left( x, y, u, u_{x}, u_{y} \right) = 0.
	    \end{equation}

	    Уравнению соответствует квадратичная форма

	    \begin{equation}
	        f \left( t_{1}, t_{2} \right) = A t^{2}_{1} + 2 B t_{1} t_{2} + C t^{2}_{2}.
	    \end{equation}

	    Дифференциальное уравнение принадлежит:

	    \begin{enumerate}
	        \item гиперболическому типу, если $B^{2} - AC > 0$ (квадратичная форма знакопеременная);
	        \item параболическому типу, если $B^{2} - AC = 0$ (квадратичная форма знакопостоянная)
	        \item эллиптическому типу, если $B^{2} - AC < 0$ (квадратичная форма знакоопределенная)
	    \end{enumerate}

	    Введем вместо $\left( x, y \right)$ новые независимые переменные $\left( \xi \left( x, y \right), \eta \left( x, y \right) \right)$, причем якобиан

	    \begin{equation}
	        \frac{D \left( \xi, \eta \right)}{D \left( x, y \right)} \neq 0
	    \end{equation}

	    в области $D$.

	    Функции $\xi \left( x, y \right)$ и $\eta \left( x, y \right)$ можно выбрать так, чтобы исходное уравнение приняло наиболее простой вид.

	    \subsubsection*{$\mathbf{B^{2} - AC > 0}$}

	        В рассматриваемой области исходное уравнение принадлежит гиперболическому типу.\\

	        Функции $\xi \left( x, y \right)$ и $\eta \left( x, y \right)$ можно найти как интегралы следующих дифференциальных уравнений

	        \begin{equation}
	            A \frac{\partial \varphi_{1}}{\partial x} + \left( B + \sqrt{B^{2} - A C} \right) \frac{\partial \varphi_{1}}{\partial y} = 0,
	        \end{equation}

	        \begin{equation}
	            A \frac{\partial \varphi_{2}}{\partial x} + \left( B - \sqrt{B^{2} - A C} \right) \frac{\partial \varphi_{2}}{\partial y} = 0.
	        \end{equation}

	        Для интегрирования этих уравнений необходимо проинтегрировать соответствующие им системы обыкновенных дифференциальных уравнений

	        \begin{equation}
		        \frac{d x}{A} = \frac{dy}{B + \sqrt{B^{2} - A C}}, \; \frac{d x}{A} = \frac{dy}{B - \sqrt{B^{2} - A C}}.
	        \end{equation}

	        Теперь положим

	        \begin{equation}
		        \xi = \varphi_{1} \left( x, y \right), \eta = \varphi_{2} \left( x, y \right).
	        \end{equation}

			В новыйх переменных уравнение примет седующий вид:

			\begin{equation}
				u_{\xi \eta} = F_{1} \left( \xi, \eta, u, u_{\xi}, u_{\eta} \right).
			\end{equation}

		\subsubsection*{$\mathbf{B^{2} - AC = 0}$}

			В рассматриваемой области уравнение принадлежит параболическому типу.\\

			Первая переменная находится из уравнения:

			\begin{equation}
				A \frac{\partial \varphi}{\partial x} + B \frac{\partial \varphi}{\partial y} = 0 \Rightarrow \xi = \varphi \left( x, y \right).
			\end{equation}

			Вторую переменную можно найти из условия невырожденности преобразования

			\begin{equation}
				\frac{D \left( \xi, \eta \right)}{D \left( x, y \right)} =
				\begin{bmatrix}
					\xi_{x} & \xi_{y}\\
					\eta_{x} & \eta_{y}
				\end{bmatrix} =
				f \left( x, y \right) \neq 0.
			\end{equation}

			В новыйх переменных уравнение примет седующий вид:
			
			\begin{equation}
				u_{\eta \eta} = F_{2} \left( \xi, \eta, u, u_{\xi}, u_{\eta} \right).
			\end{equation}

		\subsubsection*{$\mathbf{B^{2} - AC < 0}$}

			В рассматриваемой области принадлежит эллиптическому типу.\\

			Первая переменная находится из уравнения:

		    \begin{equation}
				A \frac{\partial \varphi}{\partial x} + \left( B + i \sqrt{A C - B^{2}} \right) \frac{\partial \varphi}{\partial y} = 0 \Rightarrow \xi = \varphi \left( x, y \right).
			\end{equation}

			Вторую полагаем комплесно сопряженной первой

			\begin{equation}
				\eta = \varphi^{*} \left( x, y \right).
			\end{equation}

			 Чтобы не иметь дела с комплексными переменными, введемновые переменные $\alpha$ и $\beta$, равные

			\begin{equation}
				\alpha = \frac{\varphi + \varphi^{*}}{2}, \; \beta = \frac{\varphi - \varphi^{*}}{2 i},
			\end{equation}

			так что

			\begin{equation}
				\xi = \alpha + i \beta, \; \eta = \alpha - i \beta.
			\end{equation}

			В новыйх переменных уравнение примет седующий вид:

			\begin{equation}
				u_{\alpha \alpha} + u_{\beta \beta} = F_{3} \left( \alpha, \beta, u, u_{\alpha}, u_{\beta} \right).
			\end{equation}

		\newpage

	\subsection*{Гиперболические системы уравнений}

		Пусть дана система уравнений в частных производных

		\begin{equation}
			\mathbf{U}_{t} + \mathbf{\nabla} \cdot \mathbf{F} = \mathbf{c}, \;
			\mathbf{U} =
			\begin{bmatrix}
				U_{1} \\
				U_{2} \\
				\vdots \\
				U_{n}
			\end{bmatrix}, \;
			\mathbf{F} =
			\begin{bmatrix}
				F_{11} & F_{12} & \ldots & F_{1m} \\
				F_{21} & F_{22} & \ldots & F_{2m} \\
				\vdots & \vdots & \ddots & \vdots \\
				F_{n1} & F_{n2} & \ldots & F_{nm}
			\end{bmatrix}, \;
			\begin{bmatrix}
				c_{1} \\
				c_{2} \\
				\vdots \\
				c_{n}
			\end{bmatrix},
		\end{equation}

		которую, зная определяющие уравнения $\mathbf{U} = \mathbf{U} \left( \mathbf{u}, \mathbf{x}, t \right)$, $\mathbf{F} = \mathbf{F} \left( \mathbf{u}, \mathbf{x}, t \right)$, можно записать в следующей форме

		\begin{equation}
			\tilde{A} \mathbf{u}_{t} + \sum_{j = 1}^{m} \tilde{B}_{j} \mathbf{u}_{x_{j}} = \mathbf{c},
		\end{equation}

		где $\tilde{A}$ и $\tilde{B}_{j}$ - квадратные матрицы размерности $(n \times n)$.

		Если матрица $\tilde{A}$ не сингулярна, то систему можно записать в следующей форме

		\begin{equation}
			\mathbf{u}_{t} + \sum_{j = 1}^{m} \tilde{A}_{j} \mathbf{u}_{x_{j}} = \mathbf{b},
			\label{eqn:1}
		\end{equation}

		Система (\ref{eqn:1}) называется гиперболической в точке $\left( \mathbf{x}, t, \mathbf{u} \right)$ если задача на собственные числа и на собственные векторы для матрицы $\mathbf{P} = \sum_{j = 1}^{m} \alpha_{j} \tilde{A}_{j}$
		
		\begin{equation}
			\left( \mathbf{P} - \lambda_{k} \mathbf{I} \right) \cdot \mathbf{r}^{k} = 0
		\end{equation}
		
		имеет решение, причем все собственные числа $\lambda_{k}$ вещественны и из правых собственных векторов $ \mathbf{r}^{k}$ можно составить базис в $\mathbf{E}^{n} \left( \mathbf{u} \right)$.

		\newpage
