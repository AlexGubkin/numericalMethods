\begin{center}
    \textbf{\huge Вариант №3}
\end{center}

\begin{flushright}
    Подготовил: \textit{Губкин А.С.}\\
    E-mail: \textit{alexshtil@gmail.com}\\
\end{flushright}

\section*{Задание №1}

    Решить СЛАУ $A \cdot \mathbf{x} = \mathbf{b}$. Найти собственные числа матрицы $A$.

    \[
        \left\{
            \begin{aligned}
                2 x_{1} + 5 x_{2} + x_{3} + x_{4} &= -8,\\
                5 x_{1} + x_{2} + 3 x_{3} + 4 x_{4} &= 15,\\
                -2 x_{1} + 4 x_{2} + 2 x_{3} - 2 x_{4} &= -6,\\
                4 x_{1} + 14 x_{2} + 3 x_{3} + 3 x_{4} &= -25.
            \end{aligned}
        \right.
    \]
    
    Необходимы знания по функциям \textbf{Maxima}: {\tt matrix(), solve(), invert(), transpose(), ., $\textasciicircum \textasciicircum$, eigenvalues() (из пакета \textit{eigen})}. Уметь задавать переменные и функции в \textbf{Maxima}. Уметь работать с массивами.

\section*{Задание №2}

	Привести к каноническому виду квадратичную форму: 

	\[
		f = 3 x^{2}_{1} + 2 x^{2}_{2} + x^{2}_{3} + 4 x_{1} x_{2} + 4 x_{2} x_{3}.
	\]

	Необходимы знания по функциям \textbf{Maxima}: {\tt matrix(), ratcoeff(), ., eigenvalues(), uniteigenvectors() (из пакета \textit{eigen}), transpose(), fullratsimp(), subst()}.

\section*{Задание №3}

	Найти экстремальные значения заданной неявно функции $z$ от переменных $x$ и $y$:

	\[
		\left( x^{2} + y^{2} + z^{2} \right)^{2} = a^{2} \left( x^{2} + y^{2} - z^{2} \right), \: a = const.
	\]

	Построить график.

	Необходимы знания по функциям \textbf{Maxima}: {\tt depends(), define(), diff(), solve(), rhs(), subst(), ratsimp(), fullratsimp(), draw3d() (из пакета \textit{draw})}.

\section*{Задание №4}

    Вычислить двойной интеграл $\iint\limits_{D} f(x,y) dx dy$ и построить область $D$.

    \[
        f(x,y) = x + 2 y, \qquad D \{ y = x, \: y = 2 x, \: x = 2, \: x = 3 \}.
    \]

    Необходимы знания по функциям \textbf{Maxima}: {\tt integrate(), implicit\_plot() (из пакета \textit{implicit\_plot})}.

%    \section*{Задание №4}
%
%    Исследовать на максимум и минимум функцию двух переменных. Построить график.
%
%    \[
%        z = 2 x^{3} - x y^{2} + 5 x^{2} + y^{2}.
%    \]
%
%    Необходимые условия экстремума функции $f(x, y, \ldots)$ в точке $A$ заключаются в выполнении в этой точке равенств: $\frac{\partial f}{\partial x} = 0, \: \frac{\partial f}{\partial x} = 0, \: \ldots$. При этом функция двух переменных $z = f(x, y)$ имеет в данной точке максимум, если $\triangle = \frac{\partial^{2} f}{\partial x^{2}} \frac{\partial^{2} f}{\partial y^{2}} - \left( \frac{\partial^{2} f}{\partial x \partial y} \right)^2 > 0$ и $\frac{\partial^{2} f}{\partial x^{2}}$ или $\frac{\partial^{2} f}{\partial y^{2}} < 0$, и минимум, если $\triangle > 0$ и $\frac{\partial^{2} f}{\partial x^{2}}$ или $\frac{\partial^{2} f}{\partial y^{2}} > 0$ (при условии непрерывности частных производных).\\
%
%    Необходимы знания по функциям \textbf{Maxima}: {\tt diff(), solve(), ratsimp(), fullratsimp(), wxplot3d()}. Необходимо уметь задавать переменные и функции в \textbf{Maxima}.

\section*{Задание №5}

    Исследовать фунцию:

    \[
        y = x^2 e^{\frac{1}{x}}.
    \]
    
    Необходимы знания по функциям \textbf{Maxima}: {\tt limit(), diff(), solve(), denom(), ratsimp(), fullratsimp(), wxplot2d()}.

\section*{Задание №6}

	Исследовать неявно заданную фунцию:

	\[
		x^{\frac{2}{3}} + y^{\frac{2}{3}} = a^{\frac{2}{3}}, \: a = const > 0.
	\]

	Необходимы знания по функциям \textbf{Maxima}: {\tt limit(), diff(), solve(), subst(), denom(), ratsimp(), fullratsimp(), wximplicit\_plot() (из пакета \textit{implicit\_plot})}.

\section*{Задание №7}
    
    Найти общее и частное решение обыкновенного дифференциального уравния. Построить график частного решения.

    \[
        x y' - y = x \tan \left( \frac{y}{x} \right); \qquad y(1) = \frac{\pi}{2}.
    \]

    Необходимы знания по функциям \textbf{Maxima}: {\tt diff(), ode2(), ic1(), ic2(), ratsimp(), fullratsimp(), wxplot2d(), implicit\_plot()}.

\section*{Задание №8}
    
    Разложить в ряд Фурье периодическую функцию $f(x)$ с периодом $T$, заданную на указанном сегменте. Привести первые 10 членов разложения. Построить графики исходной функции и первых 10-и членов разложения.

    \[
        f(x) =
            \begin{cases}
                -2 x \quad &\mbox{при} \quad -\pi \leq x \leq 0\\
                3 x \quad &\mbox{при} \quad 0 \leq x \leq \pi
            \end{cases}; \quad T = 2 \pi; \quad [-\pi, \pi].
    \]
    
    Необходимы знания по функциям \textbf{Maxima}: {\tt integrate(), sum(), if, wxplot2d(), ratsimp(), fullratsimp()}.

\section*{Задание №9}

    Найти решение типа бегущей волны нелинейнго уравнения теплопроводности: 

    \[
        w_{t} + a w_{x} =  \left( w w_{x} \right)_{x}.
    \]

    Решениями типа бегущей волны называются решения вида:

    \[
        w(x, t) = W(z), \qquad z = k x - \lambda t.
    \]

    Поиск решений типа бегущей волны проводится прямой подстановкой этого выражения в исходное уравнение.\\

    Необходимы знания по функциям \textbf{Maxima}: {\tt depends(), diff(), ratsimp(), fullratsimp(), subst(), ode2()}.

\section*{Задание №10}

	Привести к каноническому виду уравнение:

	\[
		y u_{xx} - x u_{yy} + u_{x} + y u_{y} = 0.
	\]

	Необходимы знания по функциям \textbf{Maxima}: {\tt depends(), diff(), ratsimp(), fullratsimp(), subst(), ode2()}.

\section*{Задание №11}

	Найти собственные числа и собственные векторы уравнений Эйлера:

	\begin{equation}
		\mathbf{U}_{t} + \mathbf{F}_{x} = 0, \;
		\mathbf{U} =
		\begin{bmatrix}
			\rho \\
			\rho u \\
			\rho \left( e + \frac{u^{2}}{2} \right)
		\end{bmatrix}, \;
		\mathbf{F} =
		\begin{bmatrix}
			\rho u \\
			\rho u^{2} + p \\
			u \left( \rho e + \frac{\rho u^{2}}{2} + p \right)
		\end{bmatrix}.
	\end{equation}

%	\begin{equation}
%		\mathbf{U} =
%		\begin{bmatrix}
%			\rho \\
%			\rho u \\
%			\rho \left( e + \frac{u^{2}}{2} \right)
%		\end{bmatrix},
%	\end{equation}
%
%	\begin{equation}
%		\mathbf{F} =
%		\begin{bmatrix}
%			\rho u \\
%			\rho u^{2} + p \\
%			u \left( \rho e + \frac{\rho u^{2}}{2} + p \right)
%		\end{bmatrix},
%	\end{equation}

	C уравнением состояния

	\begin{equation}
		p = \rho e \left( \gamma - 1 \right).
	\end{equation}

%	\begin{equation}
%		\begin{cases}
%			& \rho_{t} + \left( \rho u \right)_{x} = 0 \\
%			& \left( \rho u \right)_{t} + \left( \rho u^{2} + p \right)_{x} = 0 \\
%			& \left( \rho \left( e + \frac{u^{2}}{2} \right) \right)_{t} + \left( u \left( \rho e + \frac{\rho u^{2}}{2} + p \right) \right)_{x} = 0 \\
%			& p = \rho e \left( \gamma - 1 \right)
%		\end{cases}.
%	\end{equation}
	
	Необходимы знания по функциям \textbf{Maxima}: {\tt depends(), diff(), ratsimp(), fullratsimp(), subst()}.

\section*{Задание №12}

	Найти собственные числа и собственные векторы уравнений теории мелкой воды:
	
	\begin{equation}
		\mathbf{U}_{t} + \mathbf{F}_{x} + \mathbf{G}_{y} = \mathbf{S},
	\end{equation}
	
	\begin{equation}
		\mathbf{U} =
		\begin{bmatrix}
			h \\
			h u \\
			h v
		\end{bmatrix}, \;
		\mathbf{F} =
		\begin{bmatrix}
			h u \\
			h u^{2} + \frac{1}{2} g h^{2} \\
			h u v
		\end{bmatrix}, \;
		\mathbf{G} =
		\begin{bmatrix}
			h v \\
			h u v \\
			h v^{2} + \frac{1}{2} g h^{2}
		\end{bmatrix},
		\mathbf{S} =
		\begin{bmatrix}
			0 \\
			-g h b_{x} \\
			-g h b_{y}
		\end{bmatrix}.
	\end{equation}
	
	Функция $b \left( x, y \right)$ описывает профиль дна.\\
	
	Необходимы знания по функциям \textbf{Maxima}: {\tt depends(), diff(), ratsimp(), fullratsimp(), subst()}.

	\newpage