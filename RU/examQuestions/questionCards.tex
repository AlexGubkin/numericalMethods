\documentclass[14pt,a4paper]{extarticle}

\usepackage[utf8]{inputenc}
\usepackage[english,russian]{babel}
\usepackage[OT1]{fontenc}

\usepackage{amsmath}
\usepackage{amsfonts}
\usepackage{amssymb}
\usepackage{makeidx}
\usepackage{graphicx}

\usepackage[left=2cm,right=2cm,top=2cm,bottom=2cm]{geometry}

\author{Губкин Алексей Сергеевич}

\begin{document}

    \center{\textbf{Билет №1}}

    \begin{enumerate}

        \item Основные уравнения в частных производных математической физики. Классификация задач для уравнений в частных производных.

        \item Дифференциальное приближение. Диссипация и дисперсия численного решения. Фазовая и групповая скорость. Диссипация и дисперсия сеточного решения.

    \end{enumerate}

    \center{\textbf{Билет №2}}

    \begin{enumerate}

        \item Типы краевых задач для дифференциальных уравнений в частных производных.

        \item Системы гиперболических законов сохранения. Характеристики. Метод характеристик.

    \end{enumerate}

    \center{\textbf{Билет №3}}

    \begin{enumerate}

        \item Математическая классификация уравнений в частных производных второго порядка. Каноническая форма уравнений гиперболического, параболического и эллиптического типа.

        \item Соотношения Рэнкина - Гюгонио. Обобщенные решения. Энтропийное условие.

    \end{enumerate}

    \center{\textbf{Билет №4}}

    \begin{enumerate}

        \item Математическая классификация систем уравнений в частных производных первого порядка.

        \item Уравнение Бюргерса (точное решение).

    \end{enumerate}

    \center{\textbf{Билет №5}}

    \begin{enumerate}

        \item Основные способы пространственной дискретизации уравнений в частных производных. Шаблон разностной схемы.

        \item Уравнение Баклея - Леверетта (точное решение).

    \end{enumerate}

    \center{\textbf{Билет №6}}

    \begin{enumerate}

        \item Согласованность и аппроксимация численной схемы.

        \item Метод контрольного объема. Дискретизация по пространству в МКО. Монотонность численной схемы. Теорема Годунова.

    \end{enumerate}

    \center{\textbf{Билет №7}}

    \begin{enumerate}

         \item Устойчивость численной схемы. Анализ устойчивости по фон Нейману.

         \item Метод искусственной вязкости. Схемы Upwind. Warming-Beam scheme.

    \end{enumerate}

    \center{\textbf{Билет №8}}

    \begin{enumerate}

         \item Одномерное уравнение переноса. Устойчивость схем с разностью по потоку, против потока и с центральными разностями.

         \item Основные типы сеточных элементов. Обшая классификация сеток. Свойства регулярных сеток. Блочно-структурированные сетки. Композитные сетки (химеры).

    \end{enumerate}

    \center{\textbf{Билет №9}}

    \begin{enumerate}

         \item Устойчивость схемы Лакса, Лакса-Вендроффа и метода Мак-Кормака.

         \item Задачи, приводящие к системам линейных алгебраических уравнений. Задачи вычислительной линейной алгебры.

    \end{enumerate}

    \center{\textbf{Билет №10}}

    \begin{enumerate}

         \item Дифференциальное приближение. Диссипация и дисперсия численного решения. Фазовая и групповая скорость. Диссипация и дисперсия сеточного решения.

         \item Методы решения СЛАУ. Методы Якоби и Гаусса-Зейделя. Теорема о сходимости итерационного метода.

    \end{enumerate}

\end{document}