    \begin{center}
        \textbf{\huge Вариант №4}
    \end{center}

    \begin{flushright}
        Подготовил: \textit{Губкин А.С}\\
        E-mail: \textit{alexshtil@gmail.com}\\
    \end{flushright}

    \section*{Задание №1}

    Решить СЛАУ $A \cdot \mathbf{x} = \mathbf{b}$. Найти собственные числа и собственные векторы матрицы $A$.

    \[
        \left\{
            \begin{aligned}
                7 x_{1} + x_{2} + 2 x_{3} + 4 x_{4} &= 3,\\
                6 x_{1} + 2 x_{2} + x_{4} &= 6,\\
                4 x_{1} + x_{2} + x_{3} + 2 x_{4} &= 2,\\
                5 x_{1} + 3 x_{2} - 3 x_{3} + 4 x_{4} &= -18.
            \end{aligned}
        \right.
    \]
    
    Необходимы знания по функциям \textbf{Maxima}: {\tt matrix(), solve(), invert(), ., eigenvalues(), eigenvectors(), ratsimp(), fullratsimp()}. Необходимо уметь задавать переменные и функции в \textbf{Maxima}.

    \section*{Задание №2}

    Вычислить интеграл:

    \[
        \int \frac{\sqrt[3]{1 + \sqrt[4]{x}}}{\sqrt{x}} dx.
    \]

    Необходимы знания по функциям \textbf{Maxima}: {\tt integrate(), ratsimp(), fullratsimp()}.

    \section*{Задание №3}

    Вычислить двойной интеграл $\iint\limits_{D} f(x,y) dx dy$ и построить область $D$.

    \[
        f(x,y) = y \ln{x}, \qquad D \{ y = \frac{1}{x}, \: y = \sqrt{x}, \: x = 1, \: x = 2 \}.
    \]

    Необходимы знания по функциям \textbf{Maxima}: {\tt integrate(), implicit\_plot()}.

    \section*{Задание №4}

    Исследовать на максимум и минимум функцию двух переменных. Построить график.

    \[
        z = 3 \ln{x} + x y^{2} - y^{3}.
    \]

    Необходимые условия экстремума функции $f(x, y, \ldots)$ в точке $A$ заключаются в выполнении в этой точке равенств: $\frac{\partial f}{\partial x} = 0, \: \frac{\partial f}{\partial x} = 0, \: \ldots$. При этом функция двух переменных $z = f(x, y)$ имеет в данной точке максимум, если $\triangle = \frac{\partial^{2} f}{\partial x^{2}} \frac{\partial^{2} f}{\partial y^{2}} - \left( \frac{\partial^{2} f}{\partial x \partial y} \right)^2 > 0$ и $\frac{\partial^{2} f}{\partial x^{2}}$ или $\frac{\partial^{2} f}{\partial y^{2}} < 0$, и минимум, если $\triangle > 0$ и $\frac{\partial^{2} f}{\partial x^{2}}$ или $\frac{\partial^{2} f}{\partial y^{2}} > 0$ (при условии непрерывности частных производных).\\

    Необходимы знания по функциям \textbf{Maxima}: {\tt diff(), solve(), ratsimp(), fullratsimp(), wxplot3d()}. Необходимо уметь задавать переменные и функции в \textbf{Maxima}.

    \section*{Задание №5}

    Исследовать фунцию:

    \[
        y = \ln \left( \frac{1 + x}{1 - x} \right).
    \]

    Исследование рекомендуется проводить по следующий схеме:
    
    \begin{enumerate}
        \item Установить точки разразрыва. Исследовать функцию на четность, нечетность, периодичность.
        \item Найти точки максимума и минимума функции, вычислить значение функции в этих точках.
        \item Найти точки перегиба графика функции, вычислить значения функции в этих точках.
        \item Найти асимптоты графика функции. Вычислить предельные значения функции в точках, граничных для ее области существования.
        \item Построить график функции.
    \end{enumerate}
    
    Необходимы знания по функциям \textbf{Maxima}: {\tt limit(), diff(), solve(), denom(), ratsimp(), fullratsimp(), wxplot2d()}. Необходимо уметь задавать переменные и функции в \textbf{Maxima}.

    \section*{Задание №6}
    
    Найти общее и частное решение обыкновенного дифференциального уравния. Построить график частного решения.

    \[
        y'' - y' e^{y} = 0; \qquad y(0) = 0, \qquad y'(0) = 1.
    \]

    Необходимы знания по функциям \textbf{Maxima}: {\tt diff(), ode2(), ic1(), ic2(), ratsimp(), fullratsimp(), wxplot2d(), implicit\_plot()}.

    \section*{Задание №7}
    
    Разложить в ряд Фурье периодическую функцию $f(x)$ с периодом $T$, заданную на указанном сегменте. Привести первые 10 членов разложения. Построить графики исходной функции и первых 10-и членов разложения.

    \[
        f(x) =
            \begin{cases}
                - x \quad &\mbox{при} \quad -\pi \leq x \leq 0\\
                0 \quad &\mbox{при} \quad 0 \leq x \leq \pi
            \end{cases}; \quad T = 2 \pi; \quad [-\pi, \pi].
    \]

    Если функция $f(x)$ задана на сегменте $[-l,l]$, где $l$ -- произвольное число, то при выполнении на этом сегменте условий Дирихле указанная функция может быть представлена в виде суммы ряда Фурье:

    \[
        \frac{a_{0}}{2} + \sum_{m = 1}^{\infty} \left( a_{m} \cos \frac{m \pi x}{l} + b_{m} \sin \frac{m \pi x}{l} \right),
    \]

    где

    \[
        a_{0} =  \frac{1}{l} \int_{-l}^{l} f(x) dx, \qquad a_{m} = \frac{1}{l} \int_{-l}^{l} f(x) \cos \frac{m \pi x}{l} dx, \qquad b_{m} = \frac{1}{l} \int_{-l}^{l} f(x) \sin \frac{m \pi x}{l} dx.
    \]
    
    Необходимы знания по функциям \textbf{Maxima}: {\tt integrate(), sum(), if, wxplot2d(), ratsimp(), fullratsimp()}.

    \section*{Задание №8}

    Найти решение типа бегущей волны нелинейного волнового уравнения: 

    \[
        w_{tt} =  \left( w w_{x} \right)_{x}.
    \]

    Решениями типа бегущей волны называются решения вида:

    \[
        w(x, t) = W(z), \qquad z = k x - \lambda t.
    \]

    Поиск решений типа бегущей волны проводится прямой подстановкой этого выражения в исходное уравнение.\\

    Необходимы знания по функциям \textbf{Maxima}: {\tt depends(), diff(), ratsimp(), fullratsimp(), subst(), ode2()}. Необходимо уметь задавать переменные и функции в \textbf{Maxima}.