%\section{Приложение}
%
%    \subsection{Задание №2}
%
%        Необходимые условия экстремума функции $f(x, y, \ldots)$ в точке $A$ заключаются в выполнении в этой точке равенств: $\frac{\partial f}{\partial x} = 0, \: \frac{\partial f}{\partial x} = 0, \: \ldots$. При этом функция двух переменных $z = f(x, y)$ имеет в данной точке максимум, если $\triangle = \frac{\partial^{2} f}{\partial x^{2}} \frac{\partial^{2} f}{\partial y^{2}} - \left( \frac{\partial^{2} f}{\partial x \partial y} \right)^2 > 0$ и $\frac{\partial^{2} f}{\partial x^{2}}$ или $\frac{\partial^{2} f}{\partial y^{2}} < 0$, и минимум, если $\triangle > 0$ и $\frac{\partial^{2} f}{\partial x^{2}}$ или $\frac{\partial^{2} f}{\partial y^{2}} > 0$ (при условии непрерывности частных производных).\\
%
%    \subsection{Задание №7}
%
%        Если функция $f(x)$ задана на сегменте $[-l,l]$, где $l$ -- произвольное число, то при выполнении на этом сегменте условий Дирихле указанная функция может быть представлена в виде суммы ряда Фурье:
%
%        \[
%            \frac{a_{0}}{2} + \sum_{m = 1}^{\infty} \left( a_{m} \cos \frac{m \pi x}{l} + b_{m} \sin \frac{m \pi x}{l} \right),
%        \]
%
%        где
%
%        \[
%            a_{0} =  \frac{1}{l} \int_{-l}^{l} f(x) dx, \qquad a_{m} = \frac{1}{l} \int_{-l}^{l} f(x) \cos \frac{m \pi x}{l} dx, \qquad b_{m} = \frac{1}{l} \int_{-l}^{l} f(x) \sin \frac{m \pi x}{l} dx.
%        \]

\newpage

\section*{Теория}

\subsection{Квадратичные формы}

	Квадратичной формой действительных переменных $x_{1}, x_{2}, \ldots, x_{n}$ называется многочлен второй степени относительно этих переменных, не содержащий свободного члена и членов первой степени. \\

	Если $n = 2$, то 

	\[
		f \left( x_{1}, x_{2} \right) = a_{11} x^{2}_{1} + a_{12} x_{1} x_{2} + a_{22} x^{2}_{2}
	\]

	Если $n = 3$, то 

	\[
		f \left( x_{1}, x_{2}, x_{3} \right) = a_{11} x^{2}_{1} + a_{22} x^{2}_{2} + a_{33} x^{2}_{2} + 2 a_{12} x_{1} x_{2} + 2 a_{13} x_{1} x_{3} + 2 a_{23} x_{2} x_{3}
	\]

	В дальнейшем все необходимые формулировки и определения приведем для квадратичной формы трех переменных.

	Матрица

	\[
		\mathbf{A} =
		\begin{bmatrix}
			a_{11} & a_{12} & a_{13} \\
			a_{21} & a_{22} & a_{23} \\
			a_{31} & x_{32} & x_{33}
		\end{bmatrix}
	\]

	у которой $a_{ik} = a_{ki}$, называется матрицей квадратичной формы $f \left( x_{1}, x_{2}, x_{3} \right)$, а соответствующий определитель - определителем этой квадратичной формы.

	Так как $\mathbf{A}$ - симметрическая матрица, то корни $\lambda_{i}$ характеристического уравнения

	\[
		\det \left( \mathbf{A} - \lambda \mathbf{I} \right) = 0
	\]

%	\[
%		\begin{bmatrix}
%			a_{11} - \lambda & a_{12} & a_{13} \\
%			a_{21} & a_{22} - \lambda & a_{23} \\
%			a_{31} & a_{32} & a_{33} - \lambda
%		\end{bmatrix} = 0
%	\]

	являются действительными числами.

	Пусть $\mathbf{u}_{i}$ нормированные собственные векторы, соответствующие характеристическим числам $\lambda_{i}$. Векторы исходной системы координат - $\mathbf{e}_{i}$.

	Матрица

	\[
		\mathbf{B} = \left[ \mathbf{e}_{i} \cdot \mathbf{u}_{j} \right]
	\]

    является матрицей перехода от базиса $\mathbf{e}_{i}$ к базису $\mathbf{u}_{i}$. Матрица $\mathbf{A}$ в новой системе координат

    \[
        \tilde{\mathbf{A}} = \mathbf{B A B}^{T}
    \]

    Формулы преобразования координат при переходе к новому ортонормированному базису имеют вид

    \[
        \mathbf{x} = \tilde{\mathbf{A}} \tilde{\mathbf{x}}
    \]

    Преобразовав с помощью этих формул квадратичную форму $f \left( x_{1}, x_{2}, x_{3} \right)$, получаем квадратичную форму

    \[
        f \left( \tilde{x}_{1}, \tilde{x}_{2}, \tilde{x}_{3} \right) = a_{11} \tilde{x}^{2}_{1} + a_{22} \tilde{x}^{2}_{2} + a_{33} \tilde{x}^{2}_{2}
    \]

    не содержащую членов с произведениями $\tilde{x}_{1} \tilde{x}_{2}$, $\tilde{x}_{1} \tilde{x}_{3}$, $\tilde{x}_{2} \tilde{x}_{3}$.

    Принято говорить, что квадратичная форма $f \left( x_{1}, x_{2}, x_{3} \right)$ приведена к каноническому виду с помощью преобразования $\tilde{\mathbf{A}}$.

    \subsection{Экстремум функции нескольких переменных}

    \textbf{Определение экстремума.} Пусть функция $f \left( P \right) = f \left( x_{1}, \ldots, x_{n} \right)$ определена в окрестности точки $P_{0}$. Если или $f \left( P_{0} \right) > f \left( P \right)$, или $f \left( P_{0} \right) < f \left( P \right)$ при $0 < \rho \left( P_{0}, P \right) < \delta$, то говорят, что функция $f \left( P \right)$ имеет строгий \textit{экстремум} (соответственно \textit{максимум} или \textit{минимум}) в точке $P_{0}$.\\

    \textbf{Необходимое условие экстремума.} Дифференцируемая функция $f \left( P \right)$ может достигать экстремума лишь в \textit{стационарной} точке $P_{0}$, т.е. такой, что $d f \left( P_{0} \right) = 0$. Следовательно, точки экстремума функции $f \left( P \right)$ удовлетворяют системе уравнений $f'_{x_{i}} \left( x_{1}, \ldots, x_{n} \right) = 0 \: (i = 1, \ldots, n)$.\\

    \textbf{Достаточное условие экстремума.} Функция $f \left( P \right)$ в точке $P_{0}$ имеет:

    \begin{itemize}
    	\item \textit{максимум}, если $d f \left( P_{0} \right), \: d^{2} f \left( P_{0} \right) < 0$, при $\sum^{n}_{i = 1} \left| d x_{i} \right| \neq 0$.
    	\item \textit{минимум}, если $d f \left( P_{0} \right), \: d^{2} f \left( P_{0} \right) > 0$, при $\sum^{n}_{i = 1} \left| d x_{i} \right| \neq 0$.
    \end{itemize}

    Исследование знака второго дифференциала $d^{2} f \left( P_{0} \right)$ может быть проведено путем приведения соответствующей квадратичной формы к каноническому виду.\\

    В частности, для случая функции $f \left( x, y \right)$ двух независимых переменных $x$ и $y$ в стационарной точке $\left( x_{0}, y_{0} \right) \left( d f \left( x_{0}, y_{0} \right) =0 \right)$ при условии, что $D = A C - B^{2}$, где $A = f''_{xx} \left( x_{0}, y_{0} \right), B = f''_{xy} \left( x_{0}, y_{0} \right), C = f''_{yy} \left( x_{0}, y_{0} \right)$ имеем:

    \begin{enumerate}
    	\item \textit{минимум}, если $D > 0, \: A > 0 \: \left( C > 0 \right)$;
    	\item \textit{максимум}, если $D > 0, \: A < 0 \: \left( C < 0 \right)$;
    	\item \textit{отсутствие экстремума}, если $D < 0$.
    \end{enumerate}

    \textbf{Условный экстремум.} Задача определния экстремума функции $f \left( P \right) = f \left( x_{1}, \ldots, x_{n} \right)$ при наличии ряда соотношений $\varphi_{i} \left( P \right) \: \left( i = 1, \ldots, m; \: m < n \right)$ сводится к нахождению обычного экстремума для \textit{функции Лагранжа}

    \[
        L \left( P \right) = f \left( P \right) + \sum^{m}_{i = 1} \lambda_{i} \varphi_{i} \left( P \right),
    \]

    где $\lambda_{i} \: \left( i = 1, \ldots, m \right)$ -- постоянные множители. Вопрос о существовании и характере условного экстремума в простейшем случае решается на основании исследования знака второго дифференциала $d^{2} L \left( P_{0} \right)$ в стационарной точке $\left( P_{0} \right)$ функции $L \left( P \right)$ при условии, что переменные $d x_{1}, \ldots, d x_{n}$ связаны соотношениями

    \[
        \sum^{n}_{j = 1} \frac{\partial \varphi_{i}}{\partial x_{j}} d x_{j} = 0 \: \left( i = 1, \ldots, m \right).
    \]

    \textbf{Абсоютный экстремум.} Функция $f \left( P \right)$, дифференцируемая в ограниченной и замкнутой области, достигает своих наибольшего и наименьшего значений в этой области или в стационарной точке, или в граничной точке области.

%    Необходимые условия экстремума функции $f(x, y, \ldots)$ в точке $A$ заключаются в выполнении в этой точке равенств: $\frac{\partial f}{\partial x} = 0, \: \frac{\partial f}{\partial x} = 0, \: \ldots$. При этом функция двух переменных $z = f(x, y)$ имеет в данной точке максимум, если $\triangle = \frac{\partial^{2} f}{\partial x^{2}} \frac{\partial^{2} f}{\partial y^{2}} - \left( \frac{\partial^{2} f}{\partial x \partial y} \right)^2 > 0$ и $\frac{\partial^{2} f}{\partial x^{2}}$ или $\frac{\partial^{2} f}{\partial y^{2}} < 0$, и минимум, если $\triangle > 0$ и $\frac{\partial^{2} f}{\partial x^{2}}$ или $\frac{\partial^{2} f}{\partial y^{2}} > 0$ (при условии непрерывности частных производных).

\subsection{Исследование функций}

    \subsubsection{Схема элементарного исследования графика функции}
    
        \begin{enumerate}
        	\item Область определения;
        	
        	\item Область значений;
        	
        	\item Четность, нечетность функции;
        	
        	\item Характерные точки:
        	
        	\begin{enumerate}
        		\item точки пересечения графика с осями;
        		
        		\item предельные значения функции;
        		
        		\item экстремальные значения;
        		
        		\item точки перегиба др.
        	\end{enumerate}
        	
        	\item Асимптоты;
        	
        	\item Построение графика.
        \end{enumerate}

    \textbf{Экстремум:}

    $f' (x) = 0 \Rightarrow x^{0}_{i} \Rightarrow f'' (x^{0}_{i}) < 0 - \max$, $f'' (x^{0}_{i}) > 0 - \min$; если не обращающаяся в нуль первая производная четного порядка, то при $f^{(2 k)} (x^{0}_{i}) > 0 - \min$ и $f^{(2 k)} (x^{0}_{i}) < 0 - \max$; если $f^{(2 k + 1)} (x^{0}_{i}) \neq 0$, то перегиб.

    \textbf{Точка перегиба:}

    Если $f'' (x^{0}_{i}) > 0$, то функция локально выпукла вниз, если $f'' (x^{0}_{i}) < 0$, то локально вверх.

    \textbf{Асимптоты:}

    Кривая $y = f(x)$ имеет горизонтальную асимптоту $y = b$, если

    \[
        \lim_{x \rightarrow \pm \infty} f(x) = b,
    \]

вертикальную асимптоту $x = a$, если при $x \rightarrow a$, или $x \rightarrow a + 0$, или $x \rightarrow a - 0$

\[
\lim_{x \rightarrow a} f(x) = \pm \infty,
\]

\[
\lim_{x \rightarrow a + 0} f(x) = \pm \infty,
\]

\[
\lim_{x \rightarrow a - 0} f(x) = \pm \infty,
\]

наклонную асимптоту $y = k x + b$, если

\[
k_{1} = \lim_{x \rightarrow + \infty} \frac{f(x)}{x} \: \mbox{или} \: k_{2} = \lim_{x \rightarrow - \infty} \frac{f(x)}{x}
\]

\[
b_{1} = \lim_{x \rightarrow + \infty} \left[ f(x) - k_{1} x \right] \: \mbox{или} \: b_{2} = \lim_{x \rightarrow - \infty} \left[ f(x) - k_{2} x \right].
\]

\subsubsection{Графики неявно заданных функций}

Пусть функция $y = y \left( x \right)$ определяется неявно уравнением $F \left( x, y \right) = 0$.\\

\textbf{Общие свойства:}

\begin{enumerate}
	\item Если $F \left( x, y \right) = F \left( - x, y \right)$, то кривая симметрична относительно \textbf{OY};
	
	\item Если $F \left( x, y \right) = F \left( x, - y \right)$, то кривая симметрична относительно \textbf{OX};
	
	\item Если $F \left( x, y \right) = F \left( - x, - y \right)$, то кривая симметрична относительно $\left( 0, 0 \right)$;
	
	\item Если $F \left( x, y \right) = F \left( y, x \right)$, то кривая симметрична относительно биссектрисы $y = x$;
	
	\item График $F \left( x + a, y \right) = 0$ получается из  путем переноса последнего на $\left| a \right|$ по оси \textbf{OX};
	
	\item График $F \left( x, y  + b \right) = 0$ получается из  путем переноса последнего на $\left| b \right|$ по оси \textbf{OY};
	
	\item График $F \left( \frac{x}{p}, y \right)$ получается из $F \left( x, y \right)$  путем растяжения в $p$ раз по оси \textbf{OX};
	
	\item График $F \left( x, \frac{y}{q} \right)$ получается из $F \left( x, y \right)$  путем растяжения в $q$ раз по оси \textbf{OY};
\end{enumerate}

\textbf{Точки пересечения кривой $F \left( x, y \right)$ с осями координат}

\begin{enumerate}
	\item С осью \textbf{OX}:
	
	\[
	\begin{cases}
	& F \left( x, 0 \right) = 0\\
	& F \left( x, y \right) = 0
	\end{cases};
	\]
	
	\item С осью \textbf{OY}:
	
	\[
	\begin{cases}
	& F \left( 0, y \right) = 0\\
	& F \left( x, y \right) = 0
	\end{cases}.
	\]
	
\end{enumerate}

\textbf{Асимптоты кривой $F \left( x, y \right)$}

\begin{enumerate}
	\item Горизонтальная асимптота: приравниваем нулю коэффициент при старшей степени $x$, если этот коэффициент постоянен, то горизонтальных асимптот нет;
	
	\item Вертикальная асимптота: приравниваем нулю коэффициент при старшей степени $y$, если он постоянен, то вертикальных асимптот нет;
	
	\item Наклонная асимптота: заменяем $y$ в уравнении $F \left( x, y \right)$ на $y = k x + b$, затем в уравнении  приравниваем два коэффициента при старших степенях $x$. Решаем полученную систему и находим $k$ и $b$.
\end{enumerate}

\textbf{Особые точки кривой $F \left( x, y \right)$}

Точка  кривой $M \left( x_{0}, y_{0} \right)$ называется особой, если ее координаты одновременно удовлетворяют трем равенствам:

\[
\begin{cases}
& F \left( x_{0}, y_{0} \right) = 0\\
& F_{x} \left( x_{0}, y_{0} \right) = 0\\
& F_{y} \left( x_{0}, y_{0} \right) = 0
\end{cases}.
\]

Угловой коэффициент касательной в особой точке (угол входа)

\[
k = - \frac{F_{x}}{F_{y}}
\]

неопределенный.\\

Предположим, что в особой точке $M \left( x_{0}, y_{0} \right)$ $\left| F_{xx} \right| + \left| F_{xy} \right| + \left| F_{yy} \right| \neq 0$ т.е. не все частные производные второго порядка обращаются в нуль. Тогда $M \left( x_{0}, y_{0} \right)$ -- двойная точка.\\

Введем характеристику двойной точки.

\[
\vartriangle = F^{2}_{xy} \left( x_{0}, y_{0} \right) - F_{xx} \left( x_{0}, y_{0} \right) F_{yy} \left( x_{0}, y_{0} \right).
\]

Тогда:

\begin{enumerate}
	\item $\vartriangle > 0$ -- кривая имеет узловую точку;
	
	\item $\vartriangle < 0$ -- изолированная точка;
	
	\item $\vartriangle = 0$ -- тогда точка $M \left( x_{0}, y_{0} \right)$ может быть:
	
	\begin{enumerate}
		\item точкой возврата первого рода;
		
		\item точкой возврата второго рода;
		
		\item изолированной точкой;
		
		\item точкой самокасания. 
	\end{enumerate}
\end{enumerate}

\textbf{Точки экстремума}

Для определения точек подозрительных на экстремум нужно  воспользоваться представлением для углового коэффициента:

\[
k = - \frac{F_{x}}{F_{y}}.
\]

Теперь, чтобы найти на кривой точку, где касательная параллельна \textbf{OX}, надо решить систему

\[
\begin{cases}
& F \left( x, y \right) = 0\\
& F_{x} \left( x, y \right) = 0
\end{cases}.
\]

Пусть $\left( x_{1}, y_{1} \right)$ -- ее корни, причем $F'_{y} \left( x_{0}, y_{0} \right) \neq 0$, тогда $M \left( x_{1}, y_{1} \right)$ для кривой $F \left( x_{0}, y_{0} \right) = 0$ будет:

\begin{itemize}
	\item точкой $Y_{\max}$, если $F_{xx} \left( x_{1}, y_{1} \right) F_{y} \left( x_{1}, y_{1} \right) > 0$;
	
	\item точкой $Y_{\min}$, если $F_{xx} \left( x_{1}, y_{1} \right) F_{y} \left( x_{1}, y_{1} \right) < 0$.
\end{itemize}

Если требуется найти точки на кривой, где касательная параллельна \textbf{OY}, нужно решить систему

\[
\begin{cases}
& F \left( x, y \right) = 0\\
& F_{y} \left( x, y \right) = 0
\end{cases}.
\]

Точка $M \left( x_{2}, y_{2} \right)$ на кривой $F \left( x, y \right) = 0$ будет:

\begin{itemize}
	\item точкой $X_{\max}$, если $F_{yy} \left( x_{2}, y_{2} \right) F_{y} \left( x_{2}, y_{2} \right) > 0$;
	
	\item точкой $X_{\min}$, если $F_{yy} \left( x_{2}, y_{2} \right) F_{y} \left( x_{2}, y_{2} \right) < 0$.
\end{itemize} 

\textbf{Точки перегиба}

Если уравнение $F \left( x, y \right) = 0$ нельзя явно разрешить относительно $y$, то найти точки перегиба очень трудно.\\

Для алгебраической кривой $F \left( x, y \right) = 0$ точки перегиба ее находятся в местах пересечения кривой и ее гессианы:

\[
\begin{cases}
& F \left( x, y \right) = 0\\
& F_{xx} F^{2}_{y} - 2 F_{xy} F_{x} F_{y} + F_{yy} F^{2}_{x} = 0
\end{cases}.
\]

\subsection{Ряд Фурье}

Если функция $f(x)$ задана на сегменте $[-l,l]$, где $l$ -- произвольное число, то при выполнении на этом сегменте условий Дирихле (см., к примеру, википедию) указанная функция может быть представлена в виде суммы ряда Фурье:

\[
f(x) = \frac{a_{0}}{2} + \sum_{m = 1}^{\infty} \left( a_{m} \cos \frac{m \pi x}{l} + b_{m} \sin \frac{m \pi x}{l} \right),
\]

где

\[
a_{0} =  \frac{1}{l} \int_{-l}^{l} f(x) dx, \qquad a_{m} = \frac{1}{l} \int_{-l}^{l} f(x) \cos \frac{m \pi x}{l} dx, \qquad b_{m} = \frac{1}{l} \int_{-l}^{l} f(x) \sin \frac{m \pi x}{l} dx.
\]