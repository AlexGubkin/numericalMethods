\begin{center}
    \textbf{\huge Вариант №4}
\end{center}

\begin{flushright}
    Подготовил: \textit{Губкин А.С.}\\
    E-mail: \textit{alexshtil@gmail.com}\\
\end{flushright}

\section*{Задание №1}

    Решить СЛАУ $A \cdot \mathbf{x} = \mathbf{b}$. Найти собственные числа матрицы $A$.

    \[
        \left\{
            \begin{aligned}
                x_{1} + 2 x_{2} + 3 x_{3} + 2 x_{4} &= 4,\\
                2 x_{1} + 3 x_{2} + x_{3} + x_{4} &= -1,\\
                x_{1} + 4 x_{2} + 4 x_{3} + 3 x_{4} &= 3,\\
                2 x_{1} + 5 x_{2} + 3 x_{3} + x_{4} &= -3.
            \end{aligned}
        \right.
    \]

    Необходимы знания по функциям \textbf{Maxima}: {\tt matrix(), solve(), invert(), ., eigenvalues()}. Необходимо уметь задавать переменные и функции в \textbf{Maxima}.

\section*{Задание №2}

    Привести к каноническому виду квадратичную форму: 

    \[
        f = 6 x^{2}_{1} + 3 x^{2}_{2} + 3 x^{2}_{3} + 4 x_{1} x_{2} + 4 x_{1} x_{3} - 8 x_{2} x_{3}.
    \]

    Необходимы знания по функциям \textbf{Maxima}: {\tt matrix(), solve(), ratcoeff(), invert(), ., eigenvalues(), uniteigenvectors() (из пакета \textit{eigen})}. Необходимо уметь задавать переменные и функции в \textbf{Maxima}.

\section*{Задание №3}

    Вычислить двойной интеграл $\iint\limits_{D} f(x,y) dx dy$ и построить область $D$.

    \[
        f(x,y) = y \ln{x}, \qquad D \{ y = \frac{1}{x}, \: y = \sqrt{x}, \: x = 1, \: x = 2 \}.
    \]

    Необходимы знания по функциям \textbf{Maxima}: {\tt integrate(), implicit\_plot()}.

\section*{Задание №4}

    Найти точки условного экстремума следующей функции:

    \[
        z = A x^{2} + 2 B x y + C y^{2}, \; \mbox{если} \; x^{2} + y^{2} = 1; \; A,B,C = const
    \]

    Построить график.

    Необходимы знания по функциям \textbf{Maxima}: {\tt diff(), solve(), ratsimp(), fullratsimp(), wxplot3d()}. Необходимо уметь задавать переменные и функции в \textbf{Maxima}.

\section*{Задание №5}

    Исследовать фунцию:

    \[
        y = \sqrt{x^{2} + 1} - \sqrt{x^2 - 1}.
    \]

    Необходимы знания по функциям \textbf{Maxima}: {\tt limit(), diff(), solve(), denom(), ratsimp(), fullratsimp(), wxplot2d()}. Необходимо уметь задавать переменные и функции в \textbf{Maxima}.

\section*{Задание №6}

    Исследовать неявно заданную фунцию:

    \[
        x^{5} + y^{5} = x y^{2}, \: a = const > 0.
    \]

    Необходимы знания по функциям \textbf{Maxima}: {\tt limit(), diff(), solve(), subst(), denom(), ratsimp(), fullratsimp(), wximplicit\_plot() (из пакета \textit{implicit\_plot})}. Необходимо уметь задавать переменные и функции в \textbf{Maxima}.

\section*{Задание №7}

    Найти общее и частное решение обыкновенного дифференциального уравния. Построить график частного решения.

    \[
        y'' - y' e^{y} = 0; \qquad y(0) = 0, \qquad y'(0) = 1.
    \]

    Необходимы знания по функциям \textbf{Maxima}: {\tt diff(), ode2(), ic1(), ic2(), ratsimp(), fullratsimp(), wxplot2d(), implicit\_plot()}.

\section*{Задание №8}
    
    Разложить в ряд Фурье периодическую функцию $f(x)$ с периодом $T$, заданную на указанном сегменте. Привести первые 10 членов разложения. Построить графики исходной функции и первых 10-и членов разложения.

    \[
        f(x) =
            \begin{cases}
                - x \quad &\mbox{при} \quad -\pi \leq x \leq 0\\
                0 \quad &\mbox{при} \quad 0 \leq x \leq \pi
            \end{cases}; \quad T = 2 \pi; \quad [-\pi, \pi].
    \]

    Необходимы знания по функциям \textbf{Maxima}: {\tt integrate(), sum(), if, wxplot2d(), ratsimp(), fullratsimp()}.
