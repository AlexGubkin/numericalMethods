\begin{center}
    \textbf{\huge Вариант №2}
\end{center}

\begin{flushright}
    Подготовил: \textit{Губкин А.С.}\\
    E-mail: \textit{alexshtil@gmail.com}\\
\end{flushright}

\section*{Задание №1}

    Решить СЛАУ $A \cdot \mathbf{x} = \mathbf{b}$. Найти собственные числа матрицы $A$.

    \[
        \left\{
            \begin{aligned}
                2 x_{1} + x_{3} + x_{4} &= 7,\\
                3 x_{1} - x_{2} + 2 x_{3} - x_{4} &= 13,\\
                6 x_{1} + 4 x_{2} - x_{3} + 3 x_{4} &= 9,\\
                x_{1} - x_{2} + 2 x_{3} - x_{4} &= 7.
            \end{aligned}
        \right.
    \]

    Необходимы знания по функциям \textbf{Maxima}: {\tt matrix(), solve(), invert(), transpose(), ., $\textasciicircum \textasciicircum$, eigenvalues() (из пакета \textit{eigen})}. Уметь задавать переменные и функции в \textbf{Maxima}. Уметь работать с массивами.

\section*{Задание №2}

    Привести к каноническому виду квадратичную форму: 

    \[
        f = 2 x^{2}_{1} + 8 x_{1} x_{2} + 8 x^{2}_{2}.
    \]

    Необходимы знания по функциям \textbf{Maxima}: {\tt matrix(), ratcoeff(), ., eigenvalues(), uniteigenvectors() (из пакета \textit{eigen}), transpose(), fullratsimp(), subst()}.

\section*{Задание №3}

    Найти экстремальные значения заданной неявно функции $z$ от переменных $x$ и $y$:

    \[
        x^{2} + y^{2} + z^{2} - x z - y z + 2 x + 2 y + 2 z - 2 = 0.
    \]

    Построить график.

    Необходимы знания по функциям \textbf{Maxima}: {\tt depends(), define(), diff(), solve(), rhs(), subst(), ratsimp(), fullratsimp(), draw3d() (из пакета \textit{draw})}.

\section*{Задание №4}

	Вычислить двойной интеграл $\iint\limits_{D} f(x,y) dx dy$ и построить область $D$.
	
	\[
		f(x,y) = x - y, \qquad D \{ y = 2 x - 1, \: y = 2 - x^{2}, \: x = -3, \: x = 1 \}.
	\]
	
	Необходимы знания по функциям \textbf{Maxima}: {\tt integrate(), implicit\_plot() (из пакета \textit{implicit\_plot})}.

\section*{Задание №5}

    Исследовать фунцию:

    \[
        y = \ln{\frac{1 - x}{1 + x}}.
    \]

    Необходимы знания по функциям \textbf{Maxima}: {\tt limit(), diff(), solve(), denom(), ratsimp(), fullratsimp(), wxplot2d()}.

\section*{Задание №6}

    Исследовать неявно заданную фунцию:

    \[
        \left( x - a \right)^{2} \left( x^{2} + y^{2} \right) = b^{2} x^{2}, \: a, b = const > 0.
    \]

    Необходимы знания по функциям \textbf{Maxima}: {\tt limit(), diff(), solve(), subst(), denom(), ratsimp(), fullratsimp(), wximplicit\_plot() (из пакета \textit{implicit\_plot})}.

\section*{Задание №7}

    Найти общее и частное решение обыкновенного дифференциального уравния. Построить график частного решения.

    \[
        y' = 4 + \frac{y}{x} + \left( \frac{y}{x} \right)^{2}; \quad y(1) = 2.
    \]

    Необходимы знания по функциям \textbf{Maxima}: {\tt diff(), ode2(), ic1(), ic2(), ratsimp(), fullratsimp(), wxplot2d(), implicit\_plot()}.

\section*{Задание №8}

    Разложить в ряд Фурье периодическую функцию $f(x)$ с периодом $T$, заданную на указанном сегменте. Привести первые 10 членов разложения. Построить графики исходной функции и первых 10-и членов разложения.

    \[
        f(x) = e^{x}; \quad T = 2 \pi; \quad [-\pi, \pi].
    \]

    Необходимы знания по функциям \textbf{Maxima}: {\tt integrate(), sum(), if, wxplot2d(), ratsimp(), fullratsimp()}.

    \newpage