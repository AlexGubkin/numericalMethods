\documentclass[12pt,a4paper]{article}
\usepackage[utf8]{inputenc}
\usepackage[english,russian]{babel}
\usepackage[OT1]{fontenc}
\usepackage{amsmath}
\usepackage{amsfonts}
\usepackage{amssymb}
\usepackage{graphicx}
\usepackage[left=2cm,right=2cm,top=2cm,bottom=2cm]{geometry}
\usepackage{pgfplots}
\author{Gubkin A.S.}
\title{Методические рекомендации к численному решению модельных уравнений математической физики}
\begin{document}

	\maketitle

    \section*{Задача №0}

		Решить численно задачу Коши для одномерного волнового уравнения:

		\begin{equation}
			\begin{cases}
				& \frac{\partial u}{\partial t} + a \frac{\partial u}{\partial x} = 0, \\
				& u \left( x, 0 \right) =
					\begin{cases}
						& 1, \: 0.1 \leqslant x \leqslant 0.3, \\
						& 0, \: x < 0.1, x > 0.3.
					\end{cases}
			\end{cases},
		\end{equation}

        \begin{center}
        \begin{tikzpicture}
        \begin{axis}[
	        title=Initial condition,
	        xlabel={$x$},
	        ylabel={$u \left( x, 0 \right)$},
	        minor tick num=0,
	        height=0.1\paperheight, 
	        width=0.6\paperwidth,
	        xmin=0,
	        xmax=1,
	        ymin=0,
	        ymax=1.1
        ]
        \addplot[
            black,
            very thick,
            domain=0:1,
            samples=1000
        ] {(x>0.1 && x<0.3) ? 1: 0};
        \end{axis}
        \end{tikzpicture}
        \end{center}

        Введем равномерную сетку по пространству и времени с шагом $\triangle x$ и $\triangle t$ соответственно и аппроксимируем частные производные следующими аналогами:

        \begin{equation}
			\frac{\partial u}{\partial t} \approx \frac{u^{n+1}_{i} - u^{n}_{i}}{\triangle t}, \qquad
			\frac{\partial u}{\partial x} \approx \frac{u^{n}_{i} - u^{n}_{i-1}}{\triangle x},
		\end{equation}

		получим дискретный аналог
		
		\begin{equation}
			\frac{u^{n+1}_{i} - u^{n}_{i}}{\triangle t} + a \frac{u^{n}_{i} - u^{n}_{i-1}}{\triangle x} = 0,
		\end{equation}

        где $a$ - скорость распространения волны. При численном решении удобно положить $a=1$.\\

		\textbf{\large Замечание!!!}\\

		Предложенный метод является условно устойчивыми, т.е. устойчивый процесс вычисления можно получить при соблюдении определенных условий. В данном случае шаги по времени и координате должны удовлетворять следующему условию:

		\begin{equation}
				Co = a \frac{\triangle t}{\triangle x} \leq 1.
		\end{equation}

	\newpage

	\section*{Задача №1}

		Решать начально-краевую задачу для одномерного уравнения диффузии:

		\begin{equation}
			\begin{cases}
				\frac{\partial u}{\partial t} = a \frac{\partial^{2} u}{\partial x^{2}}, \\
				u \left( 0, t \right) = 400, \\
				u \left( 1, t \right) = 300, \\
				u \left( x, 0 \right) =
				    \begin{cases}
						& 400, \: x \leqslant 0.5, \\
						& 300, \: x > 0.5.
					\end{cases}
			\end{cases}
		\end{equation}

        \begin{center}
        \begin{tikzpicture}
        \begin{axis}[
	        title=Initial condition,
	        xlabel={$x$},
	        ylabel={$u \left( x, 0 \right)$},
	        minor tick num=0,
	        height=0.1\paperheight, 
	        width=0.6\paperwidth,
	        xmin=0,
	        xmax=1,
	        ymin=300,
	        ymax=410
        ]
        \addplot[
            black,
            very thick,
            domain=0:1,
            samples=1000
        ] {(x<=0.5) ? 400: 300};
        \end{axis}
        \end{tikzpicture}
        \end{center}

		где $u$ -- искомая функция, которая может принимать различный физический смысл: температура, давление, концентрация и т.д., $a$ - коэффициент температуропроводности / пьезопроводности / диффузии и т.д.

		Введем равномерную сетку по пространству и времени с шагом $\triangle x$ и $\triangle t$ соответственно и аппроксимируем частные производные следующими аналогами:

		\begin{equation}
			\begin{split}
				& \frac{\partial u}{\partial t} \approx \frac{u^{n+1}_{j} - u^{n}_{j}}{\triangle t}, \\
				& \frac{\partial^{2} u}{\partial x^{2}} \approx \frac{u^{n+1}_{j+1} - 2 u^{n+1}_{j} + u^{n+1}_{j-1}}{\triangle x^{2}},
			\end{split}
		\end{equation}

		где $n$ -- временной индекс, $j$ -- пространственный индекс.\\

		Подставим эти аппроксимации в уравнение и получим конечноразностный аналог исходного уравнения:

		\begin{equation}
			\frac{u^{n+1}_{j} - u^{n}_{j}}{\triangle t} = a \frac{u^{n+1}_{j+1} - 2 u^{n+1}_{j} + u^{n+1}_{j-1}}{\triangle x^{2}}.
		\end{equation}

		Этот конечноразностный аналог не позволяет явно выразить значение $u$ на верхнем временном слое, т.к. схема неявная. Поэтому для нахождения $u^{n+1}_{j}$ нужно привлекать дополнительные алгоритмы для решения системы линейных алгебраических уравнений (далее СЛАУ).\\
		СЛАУ назывется уравнение вида
		
		\begin{equation}
			\mathbf{A} \cdot \vec{\mathbf{x}} = \vec{\mathbf{b}},
		\end{equation}

		где $\mathbf{A}$ -- матрица коэффициентов СЛАУ или просто матрица СЛАУ, $\vec{\mathbf{x}}$ -- вектор искомых величин (в данном случае вектор, состоящий из значений искомой функции в узлах сетки), $\vec{\mathbf{b}}$ -- вектор свободных членов.\\

		В алгебраической форме СЛАУ из $N$ уравнений будет иметь вид:

		\begin{equation}
			\begin{cases}
				a_{11} x_{1} + a_{12} x_{2} + a_{13} x_{3} + \cdots + a_{1N} x_{N} = b_{1} \\
				a_{21} x_{1} + a_{22} x_{2} + a_{23} x_{3} + \cdots + a_{2N} x_{N} = b_{2} \\
				\cdots \\
				a_{N1} x_{1} + a_{N2} x_{2} + a_{N3} x_{3} + \cdots + a_{NN} x_{N} = b_{N}
			\end{cases}.
		\end{equation}

		Составим матрицу $\mathbf{A}$ и вектор $\vec{\mathbf{b}}$ для нашего конечноразностного аналога на $n+1$-ом временном слое. Предположим, что пространство разбито на $N$ отрезков одинакового размера, перепишем конечноразностное уравнение:

		\begin{equation}
			- a \frac{\triangle t}{\triangle x^{2}} u^{n+1}_{j-1} + \left( 1 + 2 a \frac{\triangle t}{\triangle x^{2}} \right) u^{n+1}_{j} - a \frac{\triangle t}{\triangle x^{2}} u^{n+1}_{j+1} = u^{n}_{j}.
		\end{equation}

		Теперь, записывая последовательно каждое уравнение для $j=1 \div N - 1$ получим СЛАУ:

		\begin{equation}
			\begin{cases}
				\left( 1 + 2 a \frac{\triangle t}{\triangle x^{2}} \right) u^{n+1}_{1} - a \frac{\triangle t}{\triangle x^{2}} u^{n+1}_{2} &= u^{n}_{1} + a \frac{\triangle t}{\triangle x^{2}} u^{n+1}_{0} \\
				- a \frac{\triangle t}{\triangle x^{2}} u^{n+1}_{1} + \left( 1 + 2 a \frac{\triangle t}{\triangle x^{2}} \right) u^{n+1}_{2} - a \frac{\triangle t}{\triangle x^{2}} u^{n+1}_{3} &= u^{n}_{2} \\
				- a \frac{\triangle t}{\triangle x^{2}} u^{n+1}_{2} + \left( 1 + 2 a \frac{\triangle t}{\triangle x^{2}} \right) u^{n+1}_{3} - a \frac{\triangle t}{\triangle x^{2}} u^{n+1}_{4} &= u^{n}_{3} \\
				- a \frac{\triangle t}{\triangle x^{2}} u^{n+1}_{3} + \left( 1 + 2 a \frac{\triangle t}{\triangle x^{2}} \right) u^{n+1}_{4} - a \frac{\triangle t}{\triangle x^{2}} u^{n+1}_{5} &= u^{n}_{4} \\
				\cdots \\
				- a \frac{\triangle t}{\triangle x^{2}} u^{n+1}_{N-2} + \left( 1 + 2 a \frac{\triangle t}{\triangle x^{2}} \right) u^{n+1}_{N-1} &= u^{n}_{N-1} + a \frac{\triangle t}{\triangle x^{2}} u^{n+1}_{N}.
			\end{cases}
		\end{equation}

		Таким образом, матрица $\mathbf{A}$ и вектор свободных членов $\vec{\mathbf{b}}$ будут

		\begin{equation}
			\mathbf{A} =
			\begin{pmatrix}
				1 + 2 a \frac{\triangle t}{\triangle x^{2}} & - a \frac{\triangle t}{\triangle x^{2}} & 0 & 0 & 0 & 0 & \ldots & 0 \\
				- a \frac{\triangle t}{\triangle x^{2}} & 1 + 2 a \frac{\triangle t}{\triangle x^{2}} & - a \frac{\triangle t}{\triangle x^{2}} & 0 & 0 & 0 & \ldots & 0 \\
				0 & - a \frac{\triangle t}{\triangle x^{2}} & 1 + 2 a \frac{\triangle t}{\triangle x^{2}} & - a \frac{\triangle t}{\triangle x^{2}} & 0 & 0 & \ldots & 0 \\
				0 & 0 & - a \frac{\triangle t}{\triangle x^{2}} & 1 + 2 a \frac{\triangle t}{\triangle x^{2}} & - a \frac{\triangle t}{\triangle x^{2}} & 0 & \ldots & 0 \\
				0 & 0 & 0 & - a \frac{\triangle t}{\triangle x^{2}} & 1 + 2 a \frac{\triangle t}{\triangle x^{2}} & - a \frac{\triangle t}{\triangle x^{2}} & \ldots & 0 \\
				& & & \cdots \\
				0 & 0 & \ldots & 0 & 0 & - a \frac{\triangle t}{\triangle x^{2}} & 1 + 2 a \frac{\triangle t}{\triangle x^{2}} & - a \frac{\triangle t}{\triangle x^{2}} \\
				0 & 0 & \ldots & 0 & 0 & 0 & - a \frac{\triangle t}{\triangle x^{2}} & 1 + 2 a \frac{\triangle t}{\triangle x^{2}}
			\end{pmatrix}			
		\end{equation}

		\begin{equation}
			\vec{\mathbf{b}} =
			\begin{pmatrix}
				u^{n}_{1} + a \frac{\triangle t}{\triangle x^{2}} u^{n+1}_{0} \\
				u^{n}_{2} \\
				u^{n}_{3} \\
				u^{n}_{4} \\
				\vdots \\
				u^{n}_{N-1} + a \frac{\triangle t}{\triangle x^{2}} u^{n+1}_{N}
			\end{pmatrix}
		\end{equation}

		Для решения СЛАУ существует множество методов, прямых и итерационных. Автор данного методического пособия настоятельно рекомендует пользоваться итерационными методами, к примеру такими как метод Якоби или метод Гаусса-Зейделя.\\

		Для решения СЛАУ методом Якоби или Гаусса-Зейделя перепишем ее в эквивалентной форе:

		\begin{equation}
			\mathbf{D} \cdot \vec{\mathbf{x}} - \mathbf{E} \cdot \vec{\mathbf{x}} - \mathbf{F} \cdot \vec{\mathbf{x}} = \vec{\mathbf{b}},
		\end{equation}

		где $\mathbf{D}, \: \mathbf{E}, \: \mathbf{F}$ -- диагональная, верхняя треугольная и нижняя треугольная матрицы соответственно:\\

		\begin{equation}
			\mathbf{D} =
			\begin{pmatrix}
				d_{11}	&0		&\cdots &0		\\
				0		&d_{22}	&\ddots	&\vdots	\\
				\vdots	&\ddots	&\ddots &0		\\
				0		&\cdots	&0		&d_{NN}	\\
			\end{pmatrix},
		\end{equation}

		\begin{equation}
			\mathbf{E} =
			\begin{pmatrix}
				e_{11}	&e_{12}	&\cdots &e_{1N}			\\
				0		&e_{22}	&\ddots	&\vdots			\\
				\vdots	&\ddots	&\ddots &e_{N-1 \: N}	\\
				0		&\cdots	&0		&e_{NN}			\\
			\end{pmatrix},
		\end{equation}

		\begin{equation}
			\mathbf{F} =
			\begin{pmatrix}
				f_{11}	&0		&\cdots			&0		\\
				f_{21}	&f_{22}	&\ddots			&\vdots	\\
				\vdots	&\ddots	&\ddots			&0		\\
				f_{N1}	&\cdots	&f_{N \: N-1}	&f_{NN}	\\
			\end{pmatrix}.
		\end{equation}

		Предположим, что мы имеем некотрое $k$-е приблежение к точному решению $\vec{\mathbf{x}}$, тогда

		\begin{equation}
			\mathbf{D} \cdot \vec{\mathbf{x}}^{(k)} - \mathbf{E} \cdot \vec{\mathbf{x}}^{(k)} - \mathbf{F} \cdot \vec{\mathbf{x}}^{(k)} \neq \vec{\mathbf{b}},
		\end{equation}

		но мы можем потребовать выполнения равенства заменив одно из вхождений $\vec{\mathbf{x}}^{(k)}$ на $\vec{\mathbf{x}}^{(k+1)}$, тем самым получив итерационное сотношение. К примеру метод Якоби можно получить, заменив $\vec{\mathbf{x}}^{(k)}$ на $\vec{\mathbf{x}}^{(k+1)}$ при матрице $\mathbf{D}$:

		\begin{equation}
			\mathbf{D} \cdot \vec{\mathbf{x}}^{(k+1)} - \mathbf{E} \cdot \vec{\mathbf{x}}^{(k)} - \mathbf{F} \cdot \vec{\mathbf{x}}^{(k)} = \vec{\mathbf{b}}.
		\end{equation}

		Выразим $\vec{\mathbf{x}}^{(k+1)}$, получим:

		\begin{equation}
			\vec{\mathbf{x}}^{(k+1)} = \mathbf{D}^{-1} \mathbf{E} \cdot \vec{\mathbf{x}}^{(k)} + \mathbf{D}^{-1} \mathbf{F} \cdot \vec{\mathbf{x}}^{(k)} + \mathbf{D}^{-1} \cdot \vec{\mathbf{b}}.
		\end{equation}

		Метод Гаусса-Зейделя можно получить, заменив $\vec{\mathbf{x}}^{(k)}$ на $\vec{\mathbf{x}}^{(k+1)}$ при матрице $\mathbf{F}$:

		\begin{equation}
			\vec{\mathbf{x}}^{(k+1)} = \mathbf{F}^{-1} \mathbf{D} \cdot \vec{\mathbf{x}}^{(k)} - \mathbf{F}^{-1} \mathbf{E} \cdot \vec{\mathbf{x}}^{(k)} - \mathbf{F}^{-1} \cdot \vec{\mathbf{b}}.
		\end{equation}

		Обратный метод Гаусса-Зейделя получим, заменив $\vec{\mathbf{x}}^{(k)}$ на $\vec{\mathbf{x}}^{(k+1)}$ при матрице $\mathbf{E}$:

		\begin{equation}
			\vec{\mathbf{x}}^{(k+1)} = \mathbf{E}^{-1} \mathbf{D} \cdot \vec{\mathbf{x}}^{(k)} - \mathbf{E}^{-1} \mathbf{F} \cdot \vec{\mathbf{x}}^{(k)} - \mathbf{E}^{-1} \cdot \vec{\mathbf{b}}.
		\end{equation}

		\textbf{\large Замечания!!!}\\

		Для исходной задачи хранить матрицы $\mathbf{D}, \: \mathbf{E}, \: \mathbf{F}$ не обязательно и даже вредно. Поскольку структура матрицы $\mathbf{A}$ трехдиагональная, то можно явно использовать коэффициенты матриц в итерационном соотношении. К примеру для метода Якоби итерационное соотношение будет иметь вид:

		\begin{equation}
			u^{n+1 \: (k+1)}_{j} = \frac{1}{\left( 1 + 2 a \frac{\triangle t}{\triangle x^{2}} \right)} \left( a \frac{\triangle t}{\triangle x^{2}} u^{n+1 \: (k)}_{j+1} + a \frac{\triangle t}{\triangle x^{2}} u^{n+1 \: (k)}_{j-1} + u^{n}_{j} \right), \: j = 1 \div (N - 1),
		\end{equation}

		для метода Гаусса-Зейделя:

		\begin{equation}
			u^{n+1 \: (k+1)}_{j} = \frac{1}{\left( 1 + 2 a \frac{\triangle t}{\triangle x^{2}} \right)} \left( a \frac{\triangle t}{\triangle x^{2}} u^{n+1 \: (k)}_{j+1} + a \frac{\triangle t}{\triangle x^{2}} u^{n+1 \: (k+1)}_{j-1} + u^{n}_{j} \right), \: j = 1 \div (N - 1),
		\end{equation}

		для обратного метода Гаусса-Зейделя:

		\begin{equation}
			u^{n+1 \: (k+1)}_{j} = \frac{1}{\left( 1 + 2 a \frac{\triangle t}{\triangle x^{2}} \right)} \left( a \frac{\triangle t}{\triangle x^{2}} u^{n+1 \: (k+1)}_{j+1} + a \frac{\triangle t}{\triangle x^{2}} u^{n+1 \: (k)}_{j-1} + u^{n}_{j} \right), \: j = (N - 1) \div 1.
		\end{equation}

		Данные итерационные соотношения "крутятся" в цикле \textbf{do - while} до тех пор пока не будет достигнуто одно из условий:
	
		\begin{equation}
			\frac{\left\Vert \vec{\mathbf{x}}^{(k + 1)} - \vec{\mathbf{x}}^{(k)} \right\Vert}{\left\Vert \vec{\mathbf{x}}^{(k + 1)} + \vec{\mathbf{x}}^{(k)} \right\Vert} \leqslant \varepsilon,
		\end{equation}

		\begin{equation}
			\frac{\left\Vert \vec{\mathbf{r}}^{(k + 1)} \right\Vert}{\left\Vert \vec{\mathbf{x}}^{(k + 1)} \right\Vert} \leqslant \varepsilon,
		\end{equation}

		где $\left\Vert \vec{\mathbf{x}}^{(k + 1)} \right\Vert = \sqrt{\sum^{N-1}_{j=1} \left( u^{n+1 \: (k+1)}_{j} \right)^{2}}$ -- модуль вектора решения, $\left\Vert \vec{\mathbf{r}}^{(k + 1)} \right\Vert = \left\Vert \mathbf{A} \cdot \vec{\mathbf{x}}^{(k+1)} - \vec{\mathbf{b}}\right\Vert$ -- невязка приблеженного решения на $(k+1)$-й итерации.

		\newpage

	\section*{Задача №2}

		Решить численно задачу Коши для одномерного гиперболического закона сохранения:

		\begin{equation}
			\begin{cases}
				& \frac{\partial u}{\partial t} + \frac{\partial f \left( u \right)}{\partial x} = 0, \\
				& u \left( x, 0 \right) =
					\begin{cases}
						& 1, \: 0.1 \leqslant x \leqslant 0.3, \\
						& 0, \: x \leqslant 0.1, x \geqslant 0.3.
					\end{cases}
			\end{cases},
		\end{equation}

        \begin{center}
        \begin{tikzpicture}
        \begin{axis}[
	        title=Initial condition,
	        xlabel={$x$},
	        ylabel={$u \left( x, 0 \right)$},
	        minor tick num=0,
	        height=0.1\paperheight, 
	        width=0.6\paperwidth,
	        xmin=0,
	        xmax=1,
	        ymin=0,
	        ymax=1.1
        ]
        \addplot[
            black,
            very thick,
            domain=0:1,
            samples=1000
        ] {(x>0.1 && x<0.3) ? 1: 0};
        \end{axis}
        \end{tikzpicture}
        \end{center}

		где $u$ -- сохраняемая величина, $f \left( u \right)$ -- поток величины $u$.\\
		
		Для данной задачи положим $f \left( u \right) = \frac{u^{2}}{2}$, тогда уравнение будет называться невязким уравнением Бюргерса.\\

		Введем равномерную сетку по пространству с шагом $\triangle x$ и проинтегрируем исходное уравнение по ячейке $\triangle x \times \triangle t$:

		\begin{equation}
			\iint\limits_{\triangle x \times \triangle t} \frac{\partial u}{\partial t} dt dx + \iint\limits_{\triangle x \times \triangle t} \frac{\partial f \left( u \right)}{\partial x} dt dx = 0,
		\end{equation}

		получим дискретный аналог
		
		\begin{equation}
			\frac{u^{n+1}_{i} - u^{n}_{i}}{\triangle t} + \frac{f_{i+\frac{1}{2}} - f_{i-\frac{1}{2}}}{\triangle x} = 0,
		\end{equation}

		где $u^{n}_{i}$ -- среднее значение величины $u$ в $i$-й ячейке, $f_{i+\frac{1}{2}}$ и $f_{i-\frac{1}{2}}$ -- численные потоки величины $u$ на правой и левой грянях $i$-й ячейки соответственно.\\

		Существует множество подходов вычисления численного потока на гранях ячеек. Один из таких подходов заключается в явном вычислении численного потока через известные на данном временном слое величины. Такой подход называется метод потоков (англ. \textbf{flux method}).\\

		Простейший метод в рамках данного подхода:
		
		\begin{equation}
			\begin{split}
				&f_{i+\frac{1}{2}} = \frac{1}{2} \left( f_{i+1} + f_{i} \right) - \frac{\triangle x}{2 \triangle t} \left( u_{i+1} - u_{i} \right), \\
				&f_{i-\frac{1}{2}} = \frac{1}{2} \left( f_{i} + f_{i-1} \right) - \frac{\triangle x}{2 \triangle t} \left( u_{i} - u_{i-1} \right),
			\end{split}
		\end{equation}

		называется методом \textbf{Лакса-Фридрихса} (Lax-Friedrichs, 1954).\\
		
		Метод \textbf{Лакса-Вендрофа} (Lax-Wendroff, 1960)

		\begin{equation}
			\begin{split}
				&f_{i+\frac{1}{2}} = \frac{1}{2} \left( f_{i+1} + f_{i} \right) - \frac{1}{2} a^{2}_{i+\frac{1}{2}} \frac{\triangle t}{\triangle x} \left( u_{i+1} - u_{i} \right), \\
				&f_{i-\frac{1}{2}} = \frac{1}{2} \left( f_{i} + f_{i-1} \right) - \frac{1}{2} a^{2}_{i-\frac{1}{2}} \frac{\triangle t}{\triangle x} \left( u_{i} - u_{i-1} \right),
			\end{split}
		\end{equation}

		где величины $a_{i+\frac{1}{2}}$ и $a_{i-\frac{1}{2}}$ вычисляются следующим образом

		\begin{equation}
			\begin{split}
				&a_{i+\frac{1}{2}} =
					\begin{cases}
						&\frac{f_{i+1} - f_{i}}{u_{i+1} - u_{i}}, \: u_{i+1} \neq u_{i} \\
						&f^{\prime} \left( u_{i} \right), \: u_{i+1} = u_{i}
					\end{cases}, \\
				&a_{i-\frac{1}{2}} =
					\begin{cases}
						&\frac{f_{i} - f_{i-1}}{u_{i} - u_{i-1}}, \: u_{i} \neq u_{i-1} \\
						&f^{\prime} \left( u_{i-1} \right), \: u_{i} = u_{i-1}
					\end{cases}.
			\end{split}
		\end{equation}

		Схема \textbf{upwind}

		\begin{equation}
			\begin{split}
				&f_{i+\frac{1}{2}} =
					\begin{cases}
						&f_{i}, \: a_{i+\frac{1}{2}} \geq 0 \\
						&f_{i+1}, \: a_{i+\frac{1}{2}} < 0
					\end{cases}, \\
				&f_{i-\frac{1}{2}} =
					\begin{cases}
						&f_{i-1}, \: a_{i-\frac{1}{2}} \geq 0 \\
						&f_{i}, \: a_{i-\frac{1}{2}} < 0
					\end{cases}.
			\end{split}
		\end{equation}

		очень проста в реализации, имеет хорошую устойчивость и даёт приемлемый результат.\\

		\textbf{\large Замечание!!!}\\

		Все рассмотренные выше схемы являются условно устойчивыми, т.е. устойчивый процесс вычисления можно получить при соблюдении определенных условий. В данном случае шаги по времени и координате должны удовлетворять следующему условию:

		\begin{equation}
				Co = \max_{i} \left| a_{i} \right| \frac{\triangle t}{\triangle x} \leq 1, \: a_{i} = \left. \frac{\partial f}{\partial u} \right|_{u_{i}}
		\end{equation}

	\newpage

	\section*{Задача №3}

		Решить численно задачу Коши для одномерного гиперболического закона сохранения:

		\begin{equation}
			\begin{cases}
				& \frac{\partial u}{\partial t} + \frac{\partial}{\partial x} \left( \frac{u^{2}}{u^{2} + \frac{1}{4} \left( 1 - u^{2} \right)^{2}} \right) = 0, \\
				& u \left( x, 0 \right) =
					\begin{cases}
						& \frac{3}{4}, \: x \leq 0, \\
						& 0, \: x > 0.
					\end{cases}
			\end{cases}.
		\end{equation}

		Данная задача отличается от передыдущей только видом функции потока $f \left( u \right) = \frac{u^{2}}{u^{2} + \frac{1}{4} \left( 1 - u^{2} \right)^{2}}$, поэтому решать ее следует так же как аналогичную задачу для уравнения Бюргерса.\\

		Данное уравнение возникает в задаче двухфазной фильтрации несжимаемых флюидов без учета капиллярных сил.\\

		Запишем уравнения неразрывности для каждой фазы

        \[
                m s_{1t} + w_{1x} = 0,
        \]

        \[
                m s_{2t} + w_{2x} = 0,
        \]

		где $s_{1,2}$ -- насыщенность, $m$ -- пористость.\\

		Скорости фильтрации $w_{1,2}$ подчиняются уравнению Дарси

        \[
                w_{1} = \frac{Q_{1}}{S} = - k \frac{k_{1}(s)}{\mu_{1}} p_{x},
        \]

        \[
                w_{2} = \frac{Q_{2}}{S} = - k \frac{k_{2}(s)}{\mu_{2}} p_{x},
        \]

		где $Q_{1,2}$ -- объемные расходы через сечение $S$, $k$ -- абсолютная проницаемость, $k_{1,2}$ -- относительные фазовые прроицаемости, $\mu_{1,2}$ -- динамические вязкости, $p_{x}$ -- градиент давления.

        Предполагается, что $s = s_{1}$ (либо $s = s_{2}$).\\

        Для двухфазного течения $s_{1}$ и $s_{2}$ связаны очевидным соотношением

        \[
                s_{1} + s_{2} = 1.
        \]

        Складывая уравнения неразрывности, получим

        \[
                \left( w_{1} + w_{2} \right)_{x} = 0,
        \]

        откуда найдем интеграл

        \[
                w_{1} + w_{2} = w(t) \: \mbox{или} \: Q_{1} + Q_{2} = Q(t).
        \]

		Исключим градиент давления

        \[
                \frac{w_{1}}{w_{2}} = \frac{k_{1}(s)}{\frac{\mu_{1}}{\mu_{2}} k_{2}(s)}.
        \]

        Используя соотношение $w_{1} + w_{2} = w(t)$, получим

        \[
                \frac{w_{1}}{w(t)} = \frac{k_{1}(s)}{k_{1}(s) + \frac{\mu_{1}}{\mu_{2}} k_{2}(s)} = f(s) \: \mbox{- функция Баклея-Леверетта}.
        \]

        Поэтому

        \[
                w_{1} = f(s) w(t) \: \mbox{и} \: w_{2} = (1 - f(s)) w(t).
        \]

        Подставим последнее соотношение для $w_{1}$ в уравнение неразрывности для первой фазы, получим

        \[
                m s_{t} + w(t) f(s)_{x} = 0.
        \]

        Для случая, когда $w(t)/m = 1$

        \[
                s_{t} + f(s)_{x} = 0,
        \]

		что с точностью до обозначений совпадает с исходным уравнением.\\

        \textbf{\large Замечание!!!}\\

        Рекомендуется использовать схему \textbf{upwind}.

		\newpage

	\section*{Задача №4}

        IMPES
		\newpage

	\section*{Задача №5}

		Решить численно задачу Коши для одномерных уравнений газовой динамики:

		\begin{equation}
			\begin{cases}
				&\begin{split}
					&\frac{\partial \rho}{\partial t} + \frac{\partial \rho u}{\partial x} = 0, \\
					&\frac{\partial \rho u}{\partial t} + \frac{\partial \left( \rho u^{2} + p \right)}{\partial x} = 0, \\
					&\frac{\partial \rho e}{\partial t} + \frac{\partial \left( \rho e + p \right) u}{\partial x} = 0, \\
					&p = \left( \gamma - 1 \right) \rho \varepsilon,
				\end{split} \\
				&\rho =
					\begin{cases}
						&1, \: x \leq 0.5 \\
						&0.125, \: x > 0.5
					\end{cases}, \\
				&u =
					\begin{cases}
						&0, \: x \leq x_{0} \\
						&0, \: x > x_{0}
					\end{cases}, \\
				&p =
					\begin{cases}
						&1, \: x \leq x_{0} \\
						&0.1, \: x > x_{0}
					\end{cases}.
			\end{cases}
		\end{equation}

		Здесь $\rho$ -- это плотность, $u$ -- скорость движения газа, $e = \varepsilon + \frac{1}{2} u^{2}$ -- полная энергия единицы массы, $\varepsilon$ -- удельная внутрення энергия, $p$ -- давление, $\gamma$ -- показатель адиабаты (для воздуха равен 1.4).\\

		Преобозначим консервативные переменные и варазим функции потока и уравнение состояния через них:
	
		\begin{equation}
			\left.
				\begin{split}
					\rho &= s_{1} \\
					\rho u &= s_{2} \\
					\rho e &= s_3
				\end{split}
			\right\rbrace
			\Rightarrow
			\left\lbrace
				\begin{split}
					&\frac{\partial s_{1}}{\partial t} + \frac{\partial s_{2}}{\partial x} = 0 \\
					&\frac{\partial s_{2}}{\partial t} + \frac{\partial}{\partial x} \left( \frac{s^{2}_{2}}{s_{1}} + p \right) = 0 \\
					&\frac{\partial s_{3}}{\partial t} + \frac{\partial}{\partial x} \left( \left( s_{3} + p \right) \frac{s_{2}}{s_{1}} \right) = 0 \\
					&p = \left( \gamma - 1 \right) \left( s_{3} - \frac{s^{2}_{2}}{2 s_{1}} \right)
				\end{split}
			\right. ,
		\end{equation}

	или в векторнойформе:

	\begin{equation}
		\begin{split}
			&\frac{\partial \vec{\textbf{s}}}{\partial t} + \frac{\partial \vec{\textbf{f}} \left( \vec{\textbf{s}} \right)}{\partial x} = 0, \\
			&\vec{\textbf{s}} =
				\begin{pmatrix}
					s_{1} \\
					s_{2} \\
					s_{3}
				\end{pmatrix}, \:
			\vec{\textbf{f}} \left( \vec{\textbf{s}}\right) =
				\begin{pmatrix}
					s_{2} \\
					\frac{s^{2}_{2}}{s_{1}} + p \\
					\left( s_{3} + p \right) \frac{s_{2}}{s_{1}}
				\end{pmatrix}.			
		\end{split}
	\end{equation}

	Автор данного пособия рекомендует решать данную систему двухшаговым методом Лакса-Вендрофа поскольку это позволит избежать вычисления якобиана:\\

	\textbf{первый шаг}

	\begin{equation}
		\begin{split}
			&\vec{\textbf{s}}^{n+\frac{1}{2}}_{i+\frac{1}{2}} = \frac{1}{2} \left( \vec{\textbf{s}}^{n}_{i+1} + \vec{\textbf{s}}^{n}_{i} \right) - \frac{\triangle t}{2 \triangle x} \left( \vec{\textbf{f}} \left( \vec{\textbf{s}}^{n}_{i+1} \right) - \vec{\textbf{f}} \left( \vec{\textbf{s}}^{n}_{i} \right) \right), \\
			&\vec{\textbf{s}}^{n+\frac{1}{2}}_{i-\frac{1}{2}} = \frac{1}{2} \left( \vec{\textbf{s}}^{n}_{i} + \vec{\textbf{s}}^{n}_{i-1} \right) - \frac{\triangle t}{2 \triangle x} \left( \vec{\textbf{f}} \left( \vec{\textbf{s}}^{n}_{i} \right) - \vec{\textbf{f}} \left( \vec{\textbf{s}}^{n}_{i-1} \right) \right),
		\end{split}
	\end{equation}

	\textbf{второй шаг}

	\begin{equation}
		\vec{\textbf{s}}^{n+1}_{i} = \vec{\textbf{s}}^{n}_{i} - \frac{\triangle t}{\triangle x} \left( \vec{\textbf{f}} \left( \vec{\textbf{s}}^{n+\frac{1}{2}}_{i+\frac{1}{2}} \right) - \vec{\textbf{f}} \left( \vec{\textbf{s}}^{n+\frac{1}{2}}_{i-\frac{1}{2}} \right) \right).
	\end{equation}

	\textbf{\large Замечания!!!}\\

	Так как порядок схемы Лакса-Вендрофа второй, то в решении обязательно появятся дисперсионные высокочастотные составляющие, которые могут привести к "разваливанию" решения. Чтобы решить это проблему, на втором шаге необходимо ввести член с искусственной вязкостью:
	
	\begin{equation}
		\vec{\textbf{s}}^{n+1}_{i} = \vec{\textbf{s}}^{n}_{i} - \frac{\triangle t}{\triangle x} \left( \vec{\textbf{f}} \left( \vec{\textbf{s}}^{n+\frac{1}{2}}_{i+\frac{1}{2}} \right) - \vec{\textbf{f}} \left( \vec{\textbf{s}}^{n+\frac{1}{2}}_{i-\frac{1}{2}} \right) \right) + \mu_{a} \left( \vec{\textbf{s}}^{n}_{i+1} - 2 \vec{\textbf{s}}^{n}_{i} + \vec{\textbf{s}}^{n}_{i-1} \right).
	\end{equation}

	Искусственная вязкость $\mu_{a}$ подбирается экспериментально из серии расчетов.\\

	Схема Лакса-Вендрофа является условно устойчивой, поэтому шаг по времени должен удовлетворять условию устойчивости:

	\begin{equation}
		Co = \max_{i} \left\lbrace \left| u_{i} - c_{i} \right|, \left| u_{i} \right|, \left| u_{i} + c_{i} \right| \right\rbrace \frac{\triangle t}{\triangle x} \leq 1, \: c_{i} = \sqrt{\gamma \frac{p_{i}}{\rho_{i}}}.
	\end{equation}

\end{document}