\documentclass[12pt,a4paper]{article}
\usepackage[utf8]{inputenc}
\usepackage[english,russian]{babel}
\usepackage[OT1]{fontenc}
\usepackage[left=2cm,right=2cm,top=2cm,bottom=2cm]{geometry}
\begin{document}

	\section*{Темы курсовых работ}

	\begin{enumerate}

		\item Пределы применимости линейного закона фильтрации. Нарушение линейного закона при больших и малых скоростях.

		\item Установившееся движение несжимаемой жидкости в недеформируемой пористой среде.

		\item Плоские задачи теории фильтрации.

		\item Установившийся приток к группе совершенных скважин; метод эквивалентных фильтрационных сопротивлений.

		\item Установившееся движение однородной сжимаемой жидкости по линейному закону фильтрации.

		\item Установившееся движение однородной жидкости по нелинейному закону фильтрации.

		\item Безнапорное движение жидкости в пористой среде.

		\item Установившийся приток однородной пластовой жидкости к несовершенным скважинам.

		\item Установившийся приток газа к несовершенной скважине по линейному закону фильтрации.

		\item Особенности фильтрации неньютоновских жидкостей. Приток к несовершенной скважине.

		\item Расчёт фильтрационных сопротивлений, обусловленных несовершенством скважин и скин-эффектом.

		\item Статические задачи конусообразования. Расчёт предельных дебитов и депрессий нефтяных скважин.

		\item Статические задачи конусообразования. Расчёт предельных дебитов и депрессий газовых скважин.

		\item Дренирующие нефтяные залежи с подошвенной водой.

		\item Динамические задачи конусообразования в нефтяных и газовых залежах. Расчёт безводного периода эксплуатации и нефтеотдачи за безводный период.

		\item Совместный приток жидкостей к несовершенным скважинам.

		\item Неустановившаяся фильтрация однородной упругой жидкости и газа.

		\item Уравнения фильтрации двухфазной жидкости; теория Бакли - Леверетта.

		\item Установившееся движение газированной жидкости в пористой среде.

		\item Одномерный фильтрационный поток (три вида одномерного потока; решение задач одномерного потока; применение уравнение Лапласа).

		\item Одномерный поток в условиях нелинейных законов фильтрации (поток однородной несжимаемой жидкости; поток капельной сжимаемой жидкости и реального газа при линейном и нелинейном законах фильтрации).

		\item Установившийся приток к несовершенной скважине с прямолинейным контуром питания и эксцентрично расположенной в круговом пласте.

		\item Взаимодействие скважин кольцевой батареи.

		\item Взаимодействие скважин прямолинейной батареи (цепочки скважин).

		\item Приток жидкости к горизонтальной скважине в пласте конечной толщины.

		\item Поршневое вытеснение нефти водой при нестационарной фильтрации.

		\item Вытеснение нефти и газа водой; метод последовательной смены стационарных состояний.

		\item Вытеснение нефти газом.

		\item Упругий режим фильтрации (случай: скважина в пласте неограниченных размеров).

		\item Неустановившаяся фильтрация газа.

		\item Движение жидкостей и газов в трещиноватых и трещиновато - пористых средах.

		\item Основы моделирования процессов фильтрации нефти, газа и воды.

		\item Расчёт предельных безводных и безгазовых дебитов и депрессий несовершенных скважин дренирующих нефтегазовую залежь с подошвенной водой.

		\item Расчёт оптимального интервала вскрытия нефтенасыщенного пласта нефтегазовой залежи с подошвенной водой.

		\item Моделирование вытеснения несмешивающихся весомых жидкостей и конусообразования на щелевых лотках.

		\item Термогидродинамические задачи притока газа к несовершенным скважинам.

	\end{enumerate}
	
	\begin{thebibliography}{99}

		\bibitem{} Чарный И.А. Подземная гидродинамика.- М., ГТТИ, 1956.

		\bibitem{} Щелкачев В. Н. , Лапук Б.Б. Подземная гидравлика. ГТТИ, 1949.

		\bibitem{} Пыхачёв Г.Б., Исаев Р.Г. Подземная гидравлика. - М. - Недра, 1973.

		\bibitem{} Чарный И.А. Подземная гидродинамика. - М., ГТТИ, 1963.

		\bibitem{} Телков А.П. Подземная гидрогазодинамика. - Уфа, 1974.

		\bibitem{} Евдокимова В.А., Кочина И.Н. Сборник задач по подземной гидравлике. - М., Недра, 1979.

		\bibitem{} Чарный И. А. Основы гидродинамической теории фильтрации нефти, газа и воды. Литограф. Изд. МИНХ и ГП, 1960.

		\bibitem{} Пыхачев Г. Б. Сборник задач по курсу «Подземная гидравлика», Гостоптехзидат,1957.

		\bibitem{} Щелкачев В. Н. Проблемы педагогики высшей школы. Вариационные принципы механики. М., «Нефть и газ», 1996.

		\bibitem{} Щелкачев В. Н. Разработка нефтеводоносных пластов при упругом режиме. М., Гостоптехзидат, 1959.

		\bibitem{} Баренблатт Г. И., Ентов В. М., Рыжик В. М. Теория нестационарной фильтрации жидкости и газа. М., Недра, 1972.

		\bibitem{} Лейбензон Л. С. Собрание трудов, том. 2. М., Изд-во АН СССР,1953.

		\bibitem{} Маскет М. Физические основы технологии добычи нефти. М., Гостоптехиздат, 1953.

		\bibitem{} Маскет М. Течение однородных жидкостей в пористой среде. М., Гостоптехиздат, 1949.

	\end{thebibliography}

\end{document}